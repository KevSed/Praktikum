\section{Auswertung}
\label{sec:auswertung}

\subsection{Kalibrationsmessung der Einfachschaltung}
%
Zur Kalibrationsmessung der Einfachschaltung wird eine Sinusspannung an die
Vorverstärker gelegt. Diese besitzt dabei eine Spannungspitze von
$\SI{150}{\milli\volt}$ und ist in ihrer Frequenz variabel. Der untersuchte
Frequenzbereich verläuft von etwa $\SI{1}{\kilo\hertz}$ bis
$\SI{10}{\kilo\hertz}$. Die am Ausgang der Schaltung gemessene Spannung
$U_\text{gemessen}$ wird einheitlich durch den Vorverstärker um den Faktor
\num{1000} und durch den Gleichspannungsverstärker um den Faktor \num{10}
verstärkt, sowie zur besseren Messung variabel von einem Nachverstärker.
Aufgrund dessen muss die Spannung auf eine einheitliche Größe $U_\text{einh}$
umgerechnet werden.\\
%
Die Messwerte, sowie die berechneten, auf eine Verstärkung von \num{100}
vereinheitlichten Spannungen sind in Tabelle~\ref{tab:1fach_kalib} aufgeführt.
Aus den so bestimmten Spannungen lässt sich der Durchlasskoeffizient $\beta$
nach dem folgenden Zusammenhang \eqref{eq:beta} berechnen.
%
\begin{equation}
  \beta= \frac{U_\text{einh} \cdot \sqrt{2}}{V_g \cdot U_\text{sin}^2}
  \label{eq:beta}
\end{equation}
%
Die Gesamtverstärkung $V_g$ setzt sich hierbei wie oben bereits beschrieben aus
der Vorverstärkung von \num{1000}, sowie der Gleichspannungsverstärkung von
\num{10} und der variablen Nachverstärkung $V_N$ zusammen. Letztere ist bereits
durch die Vereinheitlichung der Spannungen berücksichtigt, sodass $V_g$ den Wert
\num{10} hat. (Ist das nun riktik? Steht so im python-script. +Fehler)
Die frequenzabhängige Verteilung des Durchlasskoeffizienten ist in
Abbildung~\ref{fig:beta} dargestellt.
%
\begin{figure}
  \includegraphics[width=0.8\textwidth]{analysis/durchlasskoeffizient.pdf}
  \caption{Normierte Durchlasskurve der Kalibrationsmessung der Einfachschaltung.}
  \label{fig:beta}
\end{figure}
%
Die Bandweite $\upD\nu$ der Frequenz resultiert aus dem Integral dieser
Verteilung. Dieses wird mit Hilfe der Trapezregel numerisch bestimmt. Es ergibt
sich:
%
\begin{equation}
  \upD\nu=\SI{3533.4}{\hertz}
\end{equation}
%
\subsection{Bestimmung des Eigenrauschens der Einfachschaltung}
%
Um die im folgenden beschriebenen Messungen möglichst frei von systematischen
Fehlern zu halten, muss das Eigenrauschen des verwendeten Verstärkers bestimmt
werden. Dieses wird anschließend aus den gemessenen Spannungen entfernt. Zur
Messung der Rauschspannungen $U_R$ wird am Vorverstärker ein Kurzschluss
eingesetzt und so bei variabler Verstärkung $V_N$ gemessen. Die Messwerte sind
in Tabelle~\ref{tab:eigenrauschen} aufgeführt. Im Folgenden sind die zur
Auswertung verwendeten Spannungen stets um das zugehörige Eigenrauschen
korrigiert.
%
\subsection{Untersuchung des thermischen Rauschens via Einfachschaltung}
%
Zur Untersuchung des in Abschnitt~\ref{sec:theorie} beschriebenen thermischen
Rauschens, wird ein variabler Widerstand mit der Einfachschaltung verbunden. Mit
Hilfe dieses Widerstandes wird das Rauschspannungsquadrat für verschiedene
Widerstände gemessen. Aus dem Verhalten dieses Rauschens lässt sich die
\textsc{Boltzmann}-Konstante $k_\text{B}$ bestimmen. [Bezug aus Theorie] \\
Die Messung wird, um einen weiteren Bereich abzudecken, für zwei verschiedene
Widerstandsbereiche durchgeführt. Dazu werden zwei variable Widerstände mit
$\SI{50}{\ohm}$ - $\SI{1000}{\ohm}$ sowie $\SI{1}{\kilo\ohm}$ -
$\SI{100}{\kilo\ohm}$ verwendet. Die erste Messung mit dem geringeren Widerstand
ergibt die in Tabelle~\ref{tab:1fach_schwach} aufgeführten Messwerte. Hierbei
ist keine Vereinheitlichung der Spannungen notwendig, da die Nachverstärkung
konstant \num{1000} beträgt. Die Messung anhand des größeren Widerstandes ergibt
die in Tabelle~\ref{tab:1fach_stark} aufgeführten Messwerte. Hier ist eine
Vereinheitlichung aufgrund der Nachverstärkung notwendig. \\
Durch lineare Ausgleichsrechnung für das mittlere Spannungsquadrat lassen sich
nun gemäß der Funktionsvorschrift
%
\begin{equation}
  \bar{U}^2(R)=m\cdot R + b
\end{equation}
%
die Parameter der linearen Funktionen bestimmen. Diese werden anschließend
verwendet, um mit Hilfe der \textsc{Nyquist}-Beziehung~\eqref{eq:nyquist} die
\textsc{Boltzmann}-Konstante zu bestimmen. Die Ergebnisse der Ausgleichsrechnung
lauten
%
\begin{align*}
  m_\text{schwach}&=\SI{26.1(1)e-5}{\volt\squared\per\ohm} \quad &b_\text{schwach}=\SI{37.3(8)e-3}{\volt\squared} \\ m_\text{stark}&=\SI{3.3(1)e-10}{\volt\squared\per\ohm} \quad &b_\text{stark}=\SI{-1.2(6)e-06}{\volt\squared}.
\end{align*}
%
Die Meswerte, sowie die Ergebnisse der Ausgleichsrechnung sind in den Abbildungen~\ref{fig:r_schwach} und \ref{fig:r_stark} dargestellt.
%
\begin{figure}
  \includegraphics[width=0.8\textwidth]{analysis/einfachschaltung_schwacher_widerstand.pdf}
  \caption{Rauschspannungsquadrate für verschiedene Widerstände des schwachen Widerstandes.}
  \label{fig:r_schwach}
\end{figure}
%
%
\begin{figure}
  \includegraphics[width=0.8\textwidth]{analysis/einfachschaltung_starker_widerstand.pdf}
  \caption{Rauschspannungsquadrate für verschiedene Widerstände des starken Widerstandes.}
  \label{fig:r_stark}
\end{figure}
%

Aus der \textsc{Nyquist}-Beziehung lässt sich die $k_B$ nun bestimmen:
%
\begin{align*}
  k_B=&\frac{\bar{U^2}}{R}\cdot\frac{1}{4\cdot T\cdot R\upD\nu} \\
     =&m\cdot2,4148\cdot10^{-7} \\
  \leadsto &k_\text{stark}= \\
  \leadsto &k_\text{schwach}=
\end{align*}
