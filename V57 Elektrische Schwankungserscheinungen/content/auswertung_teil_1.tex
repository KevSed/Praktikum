\section{Auswertung}
\label{sec:auswertung}
%
Die in der Auswertung verwendeten Mittelwerte mehrfach gemessener Größen sind gemäß der Gleichung
%
\begin{equation}
    \bar{x}=\frac{1}{n}\sum_{i=1}^n x_i
    \label{eq:mittelwert}
\end{equation}
%
bestimmt.
Die Standardabweichung des Mittelwertes ergibt sich dabei zu
%
\begin{equation}
    \symup{\Delta}\bar{x}=\sqrt{\frac{1}{n(n-1)}\sum_{i=1}^n\left(x_i-\bar{x}\right)^2}.
    \label{eq:standardabweichung}
\end{equation}
%
Resultiert eine Größe über eine Gleichung aus zwei anderen fehlerbehafteten Größen, so berechnet sich der Gesamtfehler nach der Gaußschen Fehlerfortpflanzung zu
%
\begin{equation}
    \symup{\Delta}f(x_1,x_2,...,x_n)=\sqrt{\left(\frac{\partial f}{\partial x_1}\symup{\Delta}x_1\right)^2+\left(\frac{\partial f}{\partial x_2}\symup{\Delta}x_2\right)^2+ \dotsb +\left(\frac{\partial f}{\partial x_n}\symup{\Delta}x_n\right)^2}.
    \label{eq:fehlerfortpflanzung}
\end{equation}
%
Alle in der Auswertung angegebenen Größen sind stets auf die erste signifikante Stelle des Fehlers gerundet.
Setzt sich eine Größe über mehrere Schritte aus anderen Größen zusammen, so wird erst am Ende gerundet, um Fehler zu vermeiden.
Zur Auswertung wird die Programmiersprache \texttt{python (Version 3.4.1)}
mit den Bibliothekserweiterungen \texttt{numpy}, \texttt{scipy} und \texttt{matplotlib} zur Erstellung der Grafiken und linearen Regressionen verwendet.

%
\subsection{Kalibrationsmessung der Einfachschaltung}
%
Zur Kalibrationsmessung der Einfachschaltung wird eine Sinusspannung an die
Vorverstärker gelegt. Diese besitzt dabei eine Spannungspitze von
$\SI{150}{\milli\volt}$ und ist in ihrer Frequenz variabel. Der untersuchte
Frequenzbereich verläuft von etwa $\SI{1}{\kilo\hertz}$ bis
$\SI{10}{\kilo\hertz}$. Die am Ausgang der Schaltung gemessene Spannung
$U_\text{gemessen}$ wird einheitlich durch den Vorverstärker um den Faktor
\num{1000} und durch den Gleichspannungsverstärker um den Faktor \num{10}
verstärkt, sowie zur besseren Messung variabel von einem Nachverstärker.
Aufgrund dessen muss die Spannung auf eine einheitliche Größe $U_\text{einheitlich}$
umgerechnet werden.\\
%
Die Messwerte, sowie die berechneten, auf eine Nachverstärkung von \num{1}
vereinheitlichten Spannungen sind in Tabelle~\ref{tab:1fach_kalib} aufgeführt.
Aus den so bestimmten Spannungen lässt sich der Durchlasskoeffizient $\beta$
nach dem folgenden Zusammenhang \eqref{eq:beta} berechnen.
%
\begin{equation}
  \beta= \frac{U_\text{einh}^2 \cdot 2}{V_g \cdot U_\text{sin}^2}
  \label{eq:beta}
\end{equation}
%
Mit dem Fehler
%
\begin{equation}
  \upD\beta=\frac{U_\text{einh}\cdot\upD U_\text{einh}\cdot 4}{U^2_\text{sin}}
\end{equation}
%
Die Gesamtverstärkung $V_g$ setzt sich hierbei wie oben bereits beschrieben aus
der Vorverstärkung von \num{1000}, sowie der Gleichspannungsverstärkung $V_{=}$ von
\num{10} und der variablen Nachverstärkung $V_N$ zusammen. Da alle Verstärkungen bereits
durch die Vereinheitlichung der Spannungen berücksichtigt sind, hat $V_g$ den Wert
\num{1}. Die effektive Spannung aus dem angelegten Wechselstrom ergibt sich über den Faktor $\sqrt{2}$.
Die frequenzabhängige Verteilung des Durchlasskoeffizienten ist in
Abbildung~\ref{fig:beta} dargestellt.
%
\begin{figure}
  \centering
  \includegraphics[width=0.8\textwidth]{analysis/durchlasskoeffizient.pdf}
  \caption{Normierte Durchlasskurve der Kalibrationsmessung der Einfachschaltung.}
  \label{fig:beta}
\end{figure}
%
Die Bandweite $\upD\nu$ der Frequenz resultiert aus dem Integral dieser
Verteilung. Dieses wird mit Hilfe der Trapezregel numerisch bestimmt. Es ergibt
sich:
%
\begin{equation}
  \upD\nu=\SI{3500(70)}{\hertz}\; ,
\end{equation}
%
wobei der Fehler durch die numerische Berechnung auf etwa $\SI{2}{\percent}$ geschätzt ist.

\subsection{Bestimmung des Eigenrauschens der Einfachschaltung}
%
Um die im folgenden beschriebenen Messungen möglichst frei von systematischen
Fehlern zu halten, muss das Eigenrauschen des verwendeten Verstärkers bestimmt
werden. Dieses wird anschließend aus den gemessenen Spannungen entfernt. Zur
Messung der Rauschspannungen $U_R$ wird am Vorverstärker ein Kurzschluss
eingesetzt und so bei variabler Verstärkung $V_N$ gemessen. Die Messwerte sind
in Tabelle~\ref{tab:eigenrauschen} aufgeführt. Im Folgenden sind die zur
Auswertung verwendeten Spannungen stets um das zugehörige Eigenrauschen
korrigiert.
%
\subsection{Untersuchung des thermischen Rauschens via Einfachschaltung}
%
Zur Untersuchung des in Abschnitt~\ref{sec:theorie} beschriebenen thermischen
Rauschens, wird ein variabler Widerstand mit der Einfachschaltung verbunden. Mit
Hilfe dieses Widerstandes wird das Rauschspannungsquadrat für verschiedene
Widerstände gemessen. Aus dem Verhalten dieses Rauschens lässt sich die
\textsc{Boltzmann}-Konstante $k_\text{B}$ bestimmen. Gleichung~\eqref{eq:boltzmann} verknüpft die beiden Größen. \\
Die Messung wird, um einen weiteren Bereich abzudecken, für zwei verschiedene
Widerstandsbereiche durchgeführt. Dazu werden zwei variable Widerstände mit
$\SI{50}{\ohm}$ - $\SI{1000}{\ohm}$ sowie $\SI{1}{\kilo\ohm}$ -
$\SI{100}{\kilo\ohm}$ verwendet. Die erste Messung mit dem geringeren Widerstand
ergibt die in Tabelle~\ref{tab:1fach_schwach} aufgeführten Messwerte. Hierbei
ist keine Vereinheitlichung der Spannungen notwendig, da die Nachverstärkung
konstant \num{1000} beträgt. Die Messung anhand des größeren Widerstandes ergibt
die in Tabelle~\ref{tab:1fach_stark} aufgeführten Messwerte. Hier ist eine
Vereinheitlichung aufgrund der Nachverstärkung notwendig. \\
Durch lineare Ausgleichsrechnung für das mittlere Spannungsquadrat lassen sich
nun gemäß der Funktionsvorschrift
%
\begin{equation}
  \bar{U}^2(R)=m\cdot R + b
  \label{eq:linear}
\end{equation}
%
die Parameter der linearen Funktionen bestimmen. Diese werden anschließend
verwendet, um mit Hilfe der \textsc{Nyquist}-Beziehung~\eqref{eq:boltzmann} die
\textsc{Boltzmann}-Konstante zu bestimmen. Die Ergebnisse der Ausgleichsrechnung
lauten
%
\begin{align*}
  m_\text{schwach}&=\SI{260(1)e-19}{\volt\squared\per\ohm} \quad &b_\text{schwach}=\SI{10(8)e-17}{\volt\squared} \\ m_\text{stark}&=\SI{3.3(1)e-18}{\volt\squared\per\ohm} \quad &b_\text{stark}=\SI{-10(6)e-14}{\volt\squared}.
\end{align*}
%
Die Messwerte, sowie die Ergebnisse der Ausgleichsrechnung sind in den Abbildungen~\ref{fig:r_schwach} und \ref{fig:r_stark} dargestellt.
%
\begin{figure}
  \centering
  \includegraphics[width=0.8\textwidth]{analysis/einfachschaltung_schwacher_widerstand.pdf}
  \caption{Rauschspannungsquadrate für verschiedene Widerstände des schwachen Widerstandes.}
  \label{fig:r_schwach}
\end{figure}
%
%
\begin{figure}
  \centering
  \includegraphics[width=0.8\textwidth]{analysis/einfachschaltung_starker_widerstand.pdf}
  \caption{Rauschspannungsquadrate für verschiedene Widerstände des starken Widerstandes.}
  \label{fig:r_stark}
\end{figure}
%

Aus der \textsc{Nyquist}-Beziehung~\eqref{eq:boltzmann} lässt sich nun mit Hilfe der linearen Regressionsfunktion~\ref{eq:linear}
 die \textsc{Boltzmann}-Konstante $k_B$ bestimmen:
%
\begin{align}
  k_B=&\frac{\bar{U^2}}{R}\cdot\frac{1}{4\cdot T\cdot\upD\nu} \\
  \label{eq:k}
     =&m\left[\frac{\symup{V}^2}{\symup{\Omega}}\right]\cdot2,4148\cdot10^{-7}\left[\frac{\symup{s}}{\symup{K}}\right] \\
  (\upD k_B)^2=&\left(\frac{\upD m}{4T\upD\nu}\right)^2+\left(-\frac{m\upD T}{4T^2\upD\nu}\right)^2+\left(-\frac{m\upD_{\upD\nu}}{4T{\upD\nu}^2}\right)^2
\end{align}
%
Hierbei wird die Temperatur mit $T=\SI{293(5)}{\kelvin}$ beschrieben, während $R=$. Für $k_{B}$ ergeben sich aus den beiden Messungen die folgenden Werte.
%
\begin{align}
  \text{stark:}\quad k_{B}=&\SI{8.1(3)e-24}{\joule\second} \\
  \text{schwach:}\quad k_{B}=&\SI{6.3(2)e-24}{\joule\second}
\end{align}
%
\subsection{Kalibrationsmessung der Korrelatorschaltung}
%
Die Kalibrationsmessung der Korrelatorschaltung wird wie die Kalibration der Einfachschaltung mit Hilfe einer Sinusspannung gleicher Amplitude durchgeführt. An die Eingänge der Verstärkerkette wird allerdings im Unterschied zur ersten Kalibrationsmessung ein 1:1000-Abschwächer gelegt. Die Ausgangsspannungen werden analog für variable Frequenz $f$ vermessen und sind in Tabelle~\ref{tab:kalib_korr} aufgeführt.
Der Verlauf der analog zu oben berechneten Durchlasskurve ist in Abbildung~\ref{fig:korr_beta} dargestellt.
%
\begin{figure}
  \centering
  \includegraphics[width=0.8\textwidth]{analysis/durchlasskoeffizient_korr.pdf}
  \caption{Normierte Durchlasskurve der Korrelatorschaltung.}
  \label{fig:korr_beta}
\end{figure}
%
Das Integral dieser Verteilung ergibt wiederum die Bandweite $\upD\nu$ dieser Schaltung. Sie ergibt sich numerisch mit geschätztem Fehler von $\SI{2}{\percent}$ zu:
%
\begin{equation}
  \upD\nu = \SI{900(20)}{\hertz}
\end{equation}
%
\subsection{Untersuchung des thermischen Rauschens via Korrelatorschaltung}
%
Das thermische Rauschen wird mit Hilfe eines variablen Widerstandes untersucht. Dieser wird wie oben an den Anfang der Verstärkerkette geschlossen und die Ausgangsspannungen bei variablemn Widerstand gemessen. Die Messung wird für zwei Widerstände der Messbereiche $\SI{0}{\ohm}$ - $\SI{1000}{\ohm}$, sowie $\SI{0}{\ohm}$ - $\SI{100}{\kilo\ohm}$ durchgeführt. Die Messwerte sind in Tabelle~\ref{tab:schwach_korr} und Tabelle~\ref{tab:stark_korr} aufgeführt.
Für die Verteilungen der vereinheitlichten Rauschspannungsquadrate werden lineare Regressionen durchgeführt. Die Ergebnisse sind in Abbildung~\ref{fig:korr_schwach} und Abbildung~\ref{fig:korr_stark} dargestellt.
%
\begin{figure}
  \centering
  \includegraphics[width=0.8\textwidth]{analysis/korrelatorschaltung_schwacher_widerstand.pdf}
  \caption{Verteilung des Rauschspannungsquadrates der Korrelatorschaltung für variable Widerstände des schwachen Widerstandes.}
  \label{fig:korr_schwach}
\end{figure}
%
\begin{figure}
  \centering
  \includegraphics[width=0.8\textwidth]{analysis/korrelatorschaltung_starker_widerstand.pdf}
  \caption{Verteilung des Rauschspannungsquadrates der Korrelatorschaltung für variable Widerstände des schwachen Widerstandes.}
  \label{fig:korr_stark}
\end{figure}
%
Die Ergebnisse der Regression lauten:
%
\begin{align*}
  m_\text{schwach}&=\SI{1.08(2)e-17}{\volt\squared\per\ohm} \quad &b_\text{schwach}=\SI{4(1)e-14}{\volt\squared} \\ m_\text{stark}&=\SI{1.49(3)e-15}{\volt\squared\per\ohm} \quad &b_\text{stark}=\SI{-10(2)e-12}{\volt\squared}.
\end{align*}
%
Daraus ergeben sich über den Zusammenhang~\ref{eq:k} folgende Werte für $k_\text{B}$.
%
\begin{align}
  \text{stark:}\quad k_{B}=&\SI{1.42(4)e-23}{\joule\second} \\
  \text{schwach:}\quad k_{B}=&\SI{1.02(3)e-23}{\joule\second}
\end{align}
%
\subsection{Bestimmung der Rauschzahlen}

Die Rauschzahlen der beiden Verstärkerketten beschreiben die Stärke des Eigenrauschens. Sie lassen sich nach der Formel
%
\begin{equation}
  F(Z,T)=\frac{\bar{U^2}}{4kTR\upD\nu V_\text{ges}}
\end{equation}
%
mit dem Fehler
%
\begin{align*}
  (\upD F)^2=&\left(\frac{\upD \bar{U^2}}{4kTR\upD\nu V_\text{ges}}\right)^2+\left(\frac{-\upD \bar{U^2}\upD k}{4k^2TR\upD\nu V_\text{ges}}\right)^2+\left(\frac{-\upD \bar{U^2}\upD T}{4kT^2R\upD\nu V_\text{ges}}\right)^2 \\
  &+\left(\frac{-\upD \bar{U^2}\upD(\upD\nu)}{4kTR(\upD\nu)^2 V_\text{ges}}\right)^2
\end{align*}
%
berechnen. Die Ergebnisse für beide Verstärkerketten bei einer geschätzten Temperatur von $\SI{293(5)}{\kelvin}$ und Widerständen von $R=\SI{501}{\ohm}$ für die Einfachschaltung, sowie $R=\SI{550}{\ohm}$ für die Korrelatorschaltung lauten:
%
\begin{align*}
  F_\text{einf}&=0,46\,\pm\,0,01 \\
  F_\text{korr}&=0,79\,\pm\,0,02
\end{align*}
