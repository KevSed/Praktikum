\section{Messdaten}
\label{sec:Messdaten}

\begin{table}
  \centering
  \begin{tabular}{S[table-format=1.3]
                  S[table-format=3.0]
                  S[table-format=1.3]
                  S[table-format=1.4]}
    \toprule
    {$\nu$\/\text{kHz}} & {$V_{\symup{N}}$} & {$U_{\symup{gemessen}}$} & {$U_{\symup{einh}}$} \\
    \midrule
    1.078 & 500 & 0.51  & 0.204  \\
    1.586 & 200 & 0.089 & 0.02225\\
    2.036 & 200 & 0.314 & 0.0785 \\
    2.530 & 200 & 1.44  & 0.36   \\
    3.021 & 200 & 5.06  & 1.265  \\
    3.267 & 100 & 1.99  & 1.99   \\
    3.485 & 100 & 2.86  & 2.86   \\
    3.725 & 100 & 3.89  & 3.89   \\
    3.955 & 100 & 4.79  & 4.79   \\
    4.205 & 100 & 5.51  & 5.51   \\
    4.444 & 100 & 5.86  & 5.86   \\
    4.773 & 100 & 5.78  & 5.78   \\
    4.963 & 100 & 5.48  & 5.48   \\
    5.263 & 100 & 4.77  & 4.77   \\
    5.424 & 100 & 4.32  & 4.32   \\
    5.793 & 100 & 3.29  & 3.29   \\
    5.915 & 100 & 2.97  & 2.97   \\
    6.432 & 100 & 1.856 & 1.856  \\
    6.916 & 100 & 1.169 & 1.169  \\
    7.445 & 200 & 2.93  & 0.7325 \\
    8.016 & 200 & 1.772 & 0.443  \\
    8.412 & 200 & 1.263 & 0.31575\\
    8.895 & 200 & 0.832 & 0.208  \\
    9.432 & 200 & 0.559 & 0.13925\\
    9.936 & 200 & 0.392 & 0.098  \\
    \bottomrule
  \end{tabular}
\caption{Messdaten der Kalibrationsmessung der Einfachschaltung bei einer
konstanten Vorverstärkung von \num{1000} und Gleichspannungsverstärkung von
\num{10}. Die Spannungen sind auf eine Nachverstärkung von \num{100}
vereinheilicht.}
  \label{tab:1fach_kalib}
\end{table}

\begin{table}
  \centering
  \begin{tabular}{SS}
    \toprule
    {$V_{\symup{N}}$} & {$U_{\symup{R}}$} \\
    \midrule
    1     & 3.1   \\
    2     & 3.0   \\
    5     & 3.1   \\
    10    & 3.1   \\
    20    & 3.0   \\
    50    & 3.0   \\
    100   & 3.0   \\
    200   & 3.9   \\
    500   & 11.6  \\
    1000  & 36.2  \\
    \bottomrule
  \end{tabular}
\caption{Messdaten zur Bestimmung des Eigenrauschens des verwendeten
Verstärkers. Gemessen bei einer Vorverstärkung von $V_V=1000$ und einer
Gleichspannungsverstärkung von $V_==10$.}
  \label{tab:eigenrauschen}
\end{table}

\begin{table}
  \centering
  \begin{tabular}{SSS}
    \toprule
    {$R\/\,\symup{\Omega}$} & {$V_{\symup{N}}$} & {$U_{\symup{R}}\/\,\symup{V}$}\\
    \midrule
    50  & 1000  & 0.049 \\
    102 & 1000  & 0.064 \\
    152 & 1000  & 0.078 \\
    207 & 1000  & 0.093 \\
    251 & 1000  & 0.104 \\
    303 & 1000  & 0.114 \\
    353 & 1000  & 0.129 \\
    404 & 1000  & 0.144 \\
    451 & 1000  & 0.152 \\
    501 & 1000  & 0.167 \\
    555 & 1000  & 0.185 \\
    605 & 1000  & 0.194 \\
    652 & 1000  & 0.208 \\
    704 & 1000  & 0.222 \\
    752 & 1000  & 0.237 \\
    807 & 1000  & 0.250 \\
    855 & 1000  & 0.261 \\
    908 & 1000  & 0.274 \\
    949 & 1000  & 0.283 \\
    995 & 1000  & 0.296 \\
    \bottomrule
  \end{tabular}
  \caption{Messdaten zur Bestimmung des thermischen Rauschens des schwachen Widerstandes. Gemessen bei einer Vorverstärkung von $V_V=1000$ und einer Gleichspannungsverstärkung von $V_==10$.}
  \label{tab:1fach_schwach}
\end{table}

\begin{table}
  \centering
  \begin{tabular}{SSS}
    \toprule
    {$R\/\,\symup{\Omega}$} & {$V_{\symup{N}}$} & {$U_{\symup{R}}\/\,\symup{V}$}\\
    \midrule
    1330  & 1000  & 0.384 \\
    5320  & 1000  & 1.378 \\
    10240 &  500  & 0.704 \\
    15030 &  500  & 1.003 \\
    19600 &  500  & 1.396 \\
    25100 &  500  & 1.691 \\
    30800 &  200  & 0.337 \\
    35100 &  100  & 0.173 \\
    40600 &  200  & 0.472 \\
    44700 &  200  & 0.492 \\
    50700 &  200  & 0.547 \\
    56300 &  200  & 0.714 \\
    60000 &  100  & 0.195 \\
    64800 &  200  & 0.731 \\
    70100 &  100  & 0.213 \\
    76600 &  100  & 0.262 \\
    80200 &  100  & 0.245 \\
    85700 &  100  & 0.262 \\
    91700 &  100  & 0.301 \\
    95200 &  100  & 0.336 \\
    97300 &  100  & 0.367 \\
    \bottomrule
  \end{tabular}
  \caption{Messdaten zur Bestimmung des thermischen Rauschens des starken Widerstandes. Gemessen bei einer Vorverstärkung von $V_V=1000$ und einer Gleichspannungsverstärkung von $V_==10$.}
  \label{tab:1fach_stark}
\end{table}

\begin{table}
  \centering
  \begin{tabular}{SSS}
    \toprule
    {$R\/\,\symup{\Omega}$} & {$V_{\symup{N}}$} & {$U_{\symup{R}}\/\,\symup{V}$}\\
    \midrule
    50  & 1000  & 0.049 \\
    102 & 1000  & 0.064 \\
    152 & 1000  & 0.078 \\
    207 & 1000  & 0.093 \\
    251 & 1000  & 0.104 \\
    303 & 1000  & 0.114 \\
    353 & 1000  & 0.129 \\
    404 & 1000  & 0.144 \\
    451 & 1000  & 0.152 \\
    501 & 1000  & 0.167 \\
    555 & 1000  & 0.185 \\
    605 & 1000  & 0.194 \\
    652 & 1000  & 0.208 \\
    704 & 1000  & 0.222 \\
    752 & 1000  & 0.237 \\
    807 & 1000  & 0.250 \\
    855 & 1000  & 0.261 \\
    908 & 1000  & 0.274 \\
    949 & 1000  & 0.283 \\
    995 & 1000  & 0.296 \\
    \bottomrule
  \end{tabular}
  \caption{Messdaten der Korrelatorschaltung zur Bestimmung des thermischen Rauschens des schwachen Widerstandes. Gemessen bei einer Vorverstärkung von $V_V=1000$ und einer Gleichspannungsverstärkung von $V_==10$.}
  \label{tab:schwach_korr}
\end{table}

\begin{table}
  \centering
  \begin{tabular}{SSS}
    \toprule
    {$R\/\,\symup{\Omega}$} & {$V_{\symup{N}}$} & {$U_{\symup{R}}\/\,\symup{V}$}\\
    \midrule
    1330   & 1000  & 0.384 \\
    5320   & 1000  & 1.378 \\
    10240  &  500  & 0.704 \\
    15030  &  500  & 1.003 \\
    19600  &  500  & 1.396 \\
    25100  &  500  & 1.691 \\
    30800  &  200  & 0.337 \\
    35100  &  100  & 0.173 \\
    40600  &  200  & 0.472 \\
    44700  &  200  & 0.492 \\
    50700  &  200  & 0.547 \\
    56300  &  200  & 0.714 \\
    60000  &  100  & 0.195 \\
    64800  &  200  & 0.731 \\
    70100  &  100  & 0.213 \\
    76600  &  100  & 0.262 \\
    80200  &  100  & 0.245 \\
    85700  &  100  & 0.262 \\
    91700  &  100  & 0.301 \\
    95200  &  100  & 0.336 \\
    97300  &  100  & 0.367 \\   
    \bottomrule
  \end{tabular}
  \caption{Messdaten der Korrelatorschaltung zur Bestimmung des thermischen Rauschens des starken Widerstandes. Gemessen bei einer Vorverstärkung von $V_V=1000$ und einer Gleichspannungsverstärkung von $V_==10$.}
  \label{tab:stark_korr}
\end{table}

\begin{table}
  \centering
  \begin{tabular}{SSS}
    \toprule
    {$R\/\,\symup{\Omega}$} & {$V_{\symup{N}}$} & {$U_{\symup{R}}\/\,\symup{V}$}\\
    \midrule
    50  & 1000  & 0.049 \\
    102 & 1000  & 0.064 \\
    152 & 1000  & 0.078 \\
    207 & 1000  & 0.093 \\
    251 & 1000  & 0.104 \\
    303 & 1000  & 0.114 \\
    353 & 1000  & 0.129 \\
    404 & 1000  & 0.144 \\
    451 & 1000  & 0.152 \\
    501 & 1000  & 0.167 \\
    555 & 1000  & 0.185 \\
    605 & 1000  & 0.194 \\
    652 & 1000  & 0.208 \\
    704 & 1000  & 0.222 \\
    752 & 1000  & 0.237 \\
    807 & 1000  & 0.250 \\
    855 & 1000  & 0.261 \\
    908 & 1000  & 0.274 \\
    949 & 1000  & 0.283 \\
    995 & 1000  & 0.296 \\
    \bottomrule
  \end{tabular}
  \caption{Messdaten der Korrelatorschaltung zur Bestimmung des thermischen Rauschens des schwachen Widerstandes. Gemessen bei einer Vorverstärkung von $V_V=1000$ und einer Gleichspannungsverstärkung von $V_==10$.}
  \label{tab:schwach_korr}
\end{table}

\begin{table}
  \centering
  \begin{tabular}{S[table-format=6.0]
                  S[table-format=2.0]
                  S[table-format=2.0]
                  S[table-format=1.2(2)]
                  S[table-format=3.0]}
    \toprule
    {$\nu$} & {$V_{\symup{S}}$} & {$V_{\symup{N}}$} & {$U_{\symup{gemessen}}$} & {$U_{\symup{einheitlich}}$} \\
    \midrule
    460000 &  1 & 50 & 1.55(1) & 0 \\
    400000 &  1 & 50 & 1.71(1) & 0 \\
    360000 &  1 & 50 & 1.70(1) & 0 \\
    340000 &  1 & 50 & 1.67(1) & 0 \\
    320000 &  1 & 50 & 1.68(1) & 0 \\
    280000 &  1 & 50 & 1.35(1) & 0 \\
    240000 &  1 & 50 & 1.20(1) & 0 \\
    200000 &  1 & 50 & 1.01(1) & 0 \\
    160000 &  1 & 50 & 0.78(1) & 0 \\
    120000 &  1 & 50 & 0.62(1) & 0 \\
    100000 & 10 &  5 & 0.68(1) & 0 \\
     80000 & 10 &  5 & 0.56(1) & 0 \\
     60000 & 10 & 10 & 1.70(1) & 0 \\
     40000 & 10 & 10 & 1.20(1) & 0 \\
     30000 & 10 &  5 & 0.25(1) & 0 \\
     20000 & 10 & 10 & 0.67(1) & 0 \\
     10000 & 10 &  5 & 0.14(1) & 0 \\
      9000 & 10 &  5 & 0.10(1) & 0 \\
      7000 & 10 & 10 & 0.30(1) & 0 \\
      5000 & 10 & 20 & 1.00(1) & 0 \\
      3000 & 10 & 20 & 0.70(1) & 0 \\
      2000 & 10 & 20 & 0.56(1) & 0 \\
      1000 & 10 & 20 & 0.41(1) & 0 \\
       900 & 10 & 20 & 0.36(1) & 0 \\
       700 & 10 & 20 & 0.34(1) & 0 \\
       500 & 10 & 10 & 0.08(1) & 0 \\
       300 & 10 & 20 & 0.32(1) & 0 \\
       200 & 10 & 20 & 0.30(1) & 0 \\
       100 & 10 & 20 & 0.35(1) & 0 \\
        90 & 10 & 20 & 0.40(1) & 0 \\
        70 & 10 & 20 & 0.45(1) & 0 \\
        30 & 10 & 20 & 0.52(1) & 0 \\
        20 & 10 & 20 & 0.58(1) & 0 \\
        10 & 10 & 20 & 0.65(1) & 0 \\
         9 & 10 & 50 & 2.50(1) & 0 \\
         7 & 10 & 50 & 2.20(1) & 0 \\
         5 & 10 & 50 & 2.30(1) & 0 \\
         3 & 10 &100 & 3.40(1) & 0 \\
    \bottomrule
  \end{tabular}
  \caption{Messdaten Oxydkathode.}
  \label{tab:oxydkathode}
\end{table}

\begin{table}
  \centering
  \begin{tabular}{S[table-format=6.0]
                  S[table-format=2.0]
                  S[table-format=3.0]
                  S[table-format=1.2(2)]
                  S[table-format=3.0]}
    \toprule
    {$\nu$} & {$V_{\symup{S}}$} & {$V_{\symup{N}}$} & {$U_{\symup{gemessen}}$} & {$U_{\symup{einheitlich}}$} \\
    \midrule
    460000 &  1 &  50 & 2.98(1) & 0 \\
    400000 &  1 &  50 & 3.22(1) & 0 \\
    360000 &  1 &  20 & 0.57(1) & 0 \\
    320000 &  1 &  20 & 0.54(1) & 0 \\
    280000 &  1 &  20 & 0.52(1) & 0 \\
    240000 &  1 &  50 & 2.93(1) & 0 \\
    200000 &  1 &  50 & 2.77(1) & 0 \\
    160000 &  1 &  50 & 2.10(1) & 0 \\
    120000 &  1 &  50 & 1.75(1) & 0 \\
    100000 & 10 &   5 & 2.10(1) & 0 \\
     90000 & 10 &   5 & 1.94(1) & 0 \\
     70000 & 10 &   5 & 1.57(1) & 0 \\
     50000 & 10 &   5 & 1.18(1) & 0 \\
     30000 & 10 &   5 & 0.75(1) & 0 \\
     20000 & 10 &   5 & 0.51(1) & 0 \\
     10000 & 10 &  10 & 1.02(1) & 0 \\
      9000 & 10 &  10 & 0.88(1) & 0 \\
      7000 & 10 &  10 & 0.68(1) & 0 \\
      5000 & 10 &  10 & 0.49(1) & 0 \\
      3000 & 10 &  20 & 1.28(1) & 0 \\
      2000 & 10 &  20 & 0.87(1) & 0 \\
      1000 & 10 &  20 & 0.42(1) & 0 \\
       900 & 10 &  50 & 2.02(1) & 0 \\
       700 & 10 &  50 & 1.65(1) & 0 \\
       500 & 10 &  50 & 1.23(1) & 0 \\
       300 & 10 &  50 & 0.68(1) & 0 \\
       200 & 10 & 100 & 1.98(1) & 0 \\
       100 & 10 & 100 & 0.98(1) & 0 \\
        90 & 10 & 100 & 1.08(1) & 0 \\
        70 & 10 & 100 & 0.95(1) & 0 \\
        30 & 10 & 100 & 1.20(1) & 0 \\
        20 & 10 &  50 & 0.80(1) & 0 \\
        10 & 10 &  20 & 0.65(1) & 0 \\
         5 & 10 &  20 & 1.30(1) & 0 \\
    \bottomrule
  \end{tabular}
  \caption{Messdaten Reinmetallkathode.}
  \label{tab:reinmetallkathode}
\end{table}
