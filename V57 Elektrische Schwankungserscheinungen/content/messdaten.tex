\section{Messdaten}
\label{sec:Messdaten}

\begin{table}
  \centering
  \begin{tabular}{S[table-format=1.3]
                  S[table-format=3.0]
                  S[table-format=1.3]
                  S[table-format=1.5]}
    \toprule
    {$\nu/\si{\kilo\hertz}$} & {$V_{\symup{N}}$} & {$U_{\symup{gemessen}}/\si{\volt}$} & {$U_{\symup{einheitlich}}/\si{\volt}$} \\
    \midrule
    1.078 & 500 & 0.51  & 0.204   \\
    1.586 & 200 & 0.089 & 0.02225 \\
    2.036 & 200 & 0.314 & 0.0785  \\
    2.530 & 200 & 1.44  & 0.36    \\
    3.021 & 200 & 5.06  & 1.265   \\
    3.267 & 100 & 1.99  & 1.99    \\
    3.485 & 100 & 2.86  & 2.86    \\
    3.725 & 100 & 3.89  & 3.89    \\
    3.955 & 100 & 4.79  & 4.79    \\
    4.205 & 100 & 5.51  & 5.51    \\
    4.444 & 100 & 5.86  & 5.86    \\
    4.773 & 100 & 5.78  & 5.78    \\
    4.963 & 100 & 5.48  & 5.48    \\
    5.263 & 100 & 4.77  & 4.77    \\
    5.424 & 100 & 4.32  & 4.32    \\
    5.793 & 100 & 3.29  & 3.29    \\
    5.915 & 100 & 2.97  & 2.97    \\
    6.432 & 100 & 1.856 & 1.856   \\
    6.916 & 100 & 1.169 & 1.169   \\
    7.445 & 200 & 2.93  & 0.7325  \\
    8.016 & 200 & 1.772 & 0.443   \\
    8.412 & 200 & 1.263 & 0.31575 \\
    8.895 & 200 & 0.832 & 0.208   \\
    9.432 & 200 & 0.559 & 0.13925 \\
    9.936 & 200 & 0.392 & 0.098   \\
    \bottomrule
  \end{tabular}
\caption{Messdaten der Kalibrationsmessung der Einfachschaltung bei einer
konstanten Vorverstärkung von \num{1000} und Gleichspannungsverstärkung von
\num{10}. Die Spannungen sind auf eine Nachverstärkung von \num{100}
vereinheilicht.}
  \label{tab:1fach_kalib}
\end{table}

\begin{table}
  \centering
  \begin{tabular}{S[table-format=4.0]
                  S[table-format=2.1]}
    \toprule
    {$V_{\symup{N}}$} & {$U_{\symup{R}}/\si{\milli\volt}$} \\
    \midrule
    1    &  3.1 \\
    2    &  3.0 \\
    5    &  3.1 \\
    10   &  3.1 \\
    20   &  3.0 \\
    50   &  3.0 \\
    100  &  3.0 \\
    200  &  3.9 \\
    500  & 11.6 \\
    1000 & 36.2 \\
    \bottomrule
  \end{tabular}
\caption{Messdaten zur Bestimmung des Eigenrauschens des verwendeten
Verstärkers. Gemessen bei einer Vorverstärkung von $V_V=1000$ und einer
Gleichspannungsverstärkung von $V_==10$.}
  \label{tab:eigenrauschen}
\end{table}

\begin{table}
  \centering
  \begin{tabular}{S[table-format=3.0]
                  S[table-format=4.0]
                  S[table-format=1.3]}
    \toprule
    {$R/\si{\ohm}$} & {$V_{\symup{N}}$} & {$U_{\symup{R}}/\si{\volt}$} \\
    \midrule
    50  & 1000 & 0.049 \\
    102 & 1000 & 0.064 \\
    152 & 1000 & 0.078 \\
    207 & 1000 & 0.093 \\
    251 & 1000 & 0.104 \\
    303 & 1000 & 0.114 \\
    353 & 1000 & 0.129 \\
    404 & 1000 & 0.144 \\
    451 & 1000 & 0.152 \\
    501 & 1000 & 0.167 \\
    555 & 1000 & 0.185 \\
    605 & 1000 & 0.194 \\
    652 & 1000 & 0.208 \\
    704 & 1000 & 0.222 \\
    752 & 1000 & 0.237 \\
    807 & 1000 & 0.250 \\
    855 & 1000 & 0.261 \\
    908 & 1000 & 0.274 \\
    949 & 1000 & 0.283 \\
    995 & 1000 & 0.296 \\
    \bottomrule
  \end{tabular}
  \caption{Messdaten zur Bestimmung des thermischen Rauschens des schwachen Widerstandes. Gemessen bei einer Vorverstärkung von $V_V=1000$ und einer Gleichspannungsverstärkung von $V_==10$.}
  \label{tab:1fach_schwach}
\end{table}

\begin{table}
  \centering
  \begin{tabular}{S[table-format=5.0]
                  S[table-format=4.0]
                  S[table-format=1.3]}
    \toprule
    {$R/\si{\ohm}$} & {$V_{\symup{N}}$} & {$U_{\symup{R}}/\si{\volt}$} \\
    \midrule
    1330  & 1000 & 0.384 \\
    5320  & 1000 & 1.378 \\
    10240 &  500 & 0.704 \\
    15030 &  500 & 1.003 \\
    19600 &  500 & 1.396 \\
    25100 &  500 & 1.691 \\
    30800 &  200 & 0.337 \\
    35100 &  100 & 0.173 \\
    40600 &  200 & 0.472 \\
    44700 &  200 & 0.492 \\
    50700 &  200 & 0.547 \\
    56300 &  200 & 0.714 \\
    60000 &  100 & 0.195 \\
    64800 &  200 & 0.731 \\
    70100 &  100 & 0.213 \\
    76600 &  100 & 0.262 \\
    80200 &  100 & 0.245 \\
    85700 &  100 & 0.262 \\
    91700 &  100 & 0.301 \\
    95200 &  100 & 0.336 \\
    97300 &  100 & 0.367 \\
    \bottomrule
  \end{tabular}
  \caption{Messdaten zur Bestimmung des thermischen Rauschens des starken Widerstandes. Gemessen bei einer Vorverstärkung von $V_V=1000$ und einer Gleichspannungsverstärkung von $V_==10$.}
  \label{tab:1fach_stark}
\end{table}

\begin{table}
  \centering
  \begin{tabular}{S[table-format=3.0]
                  S[table-format=4.0]
                  S[table-format=1.3]}
    \toprule
    {$R/\si{\ohm}$} & {$V_{\symup{N}}$} & {$U_{\symup{R}}/\si{\volt}$} \\
    \midrule
    50  & 1000 & 0.049 \\
    102 & 1000 & 0.064 \\
    152 & 1000 & 0.078 \\
    207 & 1000 & 0.093 \\
    251 & 1000 & 0.104 \\
    303 & 1000 & 0.114 \\
    353 & 1000 & 0.129 \\
    404 & 1000 & 0.144 \\
    451 & 1000 & 0.152 \\
    501 & 1000 & 0.167 \\
    555 & 1000 & 0.185 \\
    605 & 1000 & 0.194 \\
    652 & 1000 & 0.208 \\
    704 & 1000 & 0.222 \\
    752 & 1000 & 0.237 \\
    807 & 1000 & 0.250 \\
    855 & 1000 & 0.261 \\
    908 & 1000 & 0.274 \\
    949 & 1000 & 0.283 \\
    995 & 1000 & 0.296 \\
    \bottomrule
  \end{tabular}
  \caption{Messdaten der Korrelatorschaltung zur Bestimmung des thermischen Rauschens des schwachen Widerstandes. Gemessen bei einer Vorverstärkung von $V_V=1000$ und einer Gleichspannungsverstärkung von $V_==10$.}
  \label{tab:schwach_korr}
\end{table}

\begin{table}
  \centering
  \begin{tabular}{S[table-format=5.0]
                  S[table-format=4.0]
                  S[table-format=1.3]}
    \toprule
    {$R/\si{\ohm}$} & {$V_{\symup{N}}$} & {$U_{\symup{R}}/\si{\volt}$} \\
    \midrule
    1330  & 1000 & 0.384 \\
    5320  & 1000 & 1.378 \\
    10240 &  500 & 0.704 \\
    15030 &  500 & 1.003 \\
    19600 &  500 & 1.396 \\
    25100 &  500 & 1.691 \\
    30800 &  200 & 0.337 \\
    35100 &  100 & 0.173 \\
    40600 &  200 & 0.472 \\
    44700 &  200 & 0.492 \\
    50700 &  200 & 0.547 \\
    56300 &  200 & 0.714 \\
    60000 &  100 & 0.195 \\
    64800 &  200 & 0.731 \\
    70100 &  100 & 0.213 \\
    76600 &  100 & 0.262 \\
    80200 &  100 & 0.245 \\
    85700 &  100 & 0.262 \\
    91700 &  100 & 0.301 \\
    95200 &  100 & 0.336 \\
    97300 &  100 & 0.367 \\
    \bottomrule
  \end{tabular}
  \caption{Messdaten der Korrelatorschaltung zur Bestimmung des thermischen Rauschens des starken Widerstandes. Gemessen bei einer Vorverstärkung von $V_V=1000$ und einer Gleichspannungsverstärkung von $V_==10$.}
  \label{tab:stark_korr}
\end{table}

\begin{table}
  \centering
  \begin{tabular}{S[table-format=3.0]
                  S[table-format=4.0]
                  S[table-format=1.3]}
    \toprule
    {$R/\si{\ohm}$} & {$V_{\symup{N}}$} & {$U_{\symup{R}}/\si{\volt}$} \\
    \midrule
    50  & 1000 & 0.049 \\
    102 & 1000 & 0.064 \\
    152 & 1000 & 0.078 \\
    207 & 1000 & 0.093 \\
    251 & 1000 & 0.104 \\
    303 & 1000 & 0.114 \\
    353 & 1000 & 0.129 \\
    404 & 1000 & 0.144 \\
    451 & 1000 & 0.152 \\
    501 & 1000 & 0.167 \\
    555 & 1000 & 0.185 \\
    605 & 1000 & 0.194 \\
    652 & 1000 & 0.208 \\
    704 & 1000 & 0.222 \\
    752 & 1000 & 0.237 \\
    807 & 1000 & 0.250 \\
    855 & 1000 & 0.261 \\
    908 & 1000 & 0.274 \\
    949 & 1000 & 0.283 \\
    995 & 1000 & 0.296 \\
    \bottomrule
  \end{tabular}
  \caption{Messdaten der Korrelatorschaltung zur Bestimmung des thermischen Rauschens des schwachen Widerstandes. Gemessen bei einer Vorverstärkung von $V_V=1000$ und einer Gleichspannungsverstärkung von $V_==10$.}
  \label{tab:schwach_korr}
\end{table}

\begin{table}
  \centering
  \begin{tabular}{S[table-format=6.0]
                  S[table-format=2.0]
                  S[table-format=2.0]
                  S[table-format=1.2(2)]
                  S[table-format=2.4(4)]}
    \toprule
    {$\nu$/\si{\hertz}} & {$V_{\symup{S}}$} & {$V_{\symup{N}}$} & {$U_{\symup{gemessen}}/\si{\volt}$} & {$U_{\symup{einheitlich}}/\si{\volt}$} \\
    \midrule
    460000 &  1 & 50 & 1.55(1) & 24.80(16)   \\
    400000 &  1 & 50 & 1.71(1) & 27.36(16)   \\
    360000 &  1 & 50 & 1.70(1) & 27.20(16)   \\
    340000 &  1 & 50 & 1.67(1) & 26.72(16)   \\
    320000 &  1 & 50 & 1.68(1) & 26.88(16)   \\
    280000 &  1 & 50 & 1.35(1) & 21.60(16)   \\
    240000 &  1 & 50 & 1.20(1) & 19.20(16)   \\
    200000 &  1 & 50 & 1.01(1) & 16.16(16)   \\
    160000 &  1 & 50 & 0.78(1) & 12.48(16)   \\
    120000 &  1 & 50 & 0.62(1) &  9.92(16)   \\
    100000 & 10 &  5 & 0.68(1) & 10.88(16)   \\
     80000 & 10 &  5 & 0.56(1) &  8.96(16)   \\
     60000 & 10 & 10 & 1.70(1) &  6.80(04)   \\
     40000 & 10 & 10 & 1.20(1) &  4.800(4)   \\
     30000 & 10 &  5 & 0.25(1) &  4.00(16)   \\
     20000 & 10 & 10 & 0.67(1) &  2.680(4)   \\
     10000 & 10 &  5 & 0.14(1) &  2.24(16)   \\
      9000 & 10 &  5 & 0.10(1) &  1.60(16)   \\
      7000 & 10 & 10 & 0.30(1) &  1.200(4)   \\
      5000 & 10 & 20 & 1.00(1) &  1.000(1)   \\
      3000 & 10 & 20 & 0.70(1) &  0.700(1)   \\
      2000 & 10 & 20 & 0.56(1) &  0.560(1)   \\
      1000 & 10 & 20 & 0.41(1) &  0.410(1)   \\
       900 & 10 & 20 & 0.36(1) &  0.360(1)   \\
       700 & 10 & 20 & 0.34(1) &  0.340(1)   \\
       500 & 10 & 10 & 0.08(1) &  0.320(4)   \\
       300 & 10 & 20 & 0.32(1) &  0.320(1)   \\
       200 & 10 & 20 & 0.30(1) &  0.300(1)   \\
       100 & 10 & 20 & 0.35(1) &  0.350(1)   \\
        90 & 10 & 20 & 0.40(1) &  0.400(1)   \\
        70 & 10 & 20 & 0.45(1) &  0.450(1)   \\
        30 & 10 & 20 & 0.52(1) &  0.520(1)   \\
        20 & 10 & 20 & 0.58(1) &  0.580(1)   \\
        10 & 10 & 20 & 0.65(1) &  0.650(1)   \\
         9 & 10 & 50 & 2.50(1) &  0.4000(16) \\
         7 & 10 & 50 & 2.20(1) &  0.3520(16) \\
         5 & 10 & 50 & 2.30(1) &  0.3680(16) \\
         3 & 10 &100 & 3.40(1) &  0.1360(4)  \\
    \bottomrule
  \end{tabular}
  \caption{Messdaten zur Untersuchung einer Oxydkathode, bei einer  konstanten
  Vorverstärkung von 1000 und Gleichspannungsverstärkung von 10. Die Spannungen
  sind auf eine Nachverstärkung von 20 vereinheilicht.}
  \label{tab:oxydkathode}
\end{table}

\begin{table}
  \centering
  \begin{tabular}{S[table-format=6.0]
                  S[table-format=2.0]
                  S[table-format=3.0]
                  S[table-format=1.2(2)]
                  S[table-format=3.4(4)]}
    \toprule
    {$\nu/\si{\hertz}$} & {$V_{\symup{S}}$} & {$V_{\symup{N}}$} & {$U_{\symup{gemessen}}/\si{\volt}$} & {$U_{\symup{einheitlich}}/\si{\volt}$} \\
    \midrule
    460000 &  1 &  50 & 2.98(1) & 298(1)        \\
    400000 &  1 &  50 & 3.22(1) & 322(1)        \\
    360000 &  1 &  20 & 0.57(1) & 356.25(625)   \\
    320000 &  1 &  20 & 0.54(1) & 337.50(625)   \\
    280000 &  1 &  20 & 0.52(1) & 325.00(625)   \\
    240000 &  1 &  50 & 2.93(1) & 293(1)        \\
    200000 &  1 &  50 & 2.77(1) & 277(1)        \\
    160000 &  1 &  50 & 2.10(1) & 210(1)        \\
    120000 &  1 &  50 & 1.75(1) & 175(1)        \\
    100000 & 10 &   5 & 2.10(1) & 210(1)        \\
     90000 & 10 &   5 & 1.94(1) & 194(1)        \\
     70000 & 10 &   5 & 1.57(1) & 157(1)        \\
     50000 & 10 &   5 & 1.18(1) & 118(1)        \\
     30000 & 10 &   5 & 0.75(1) &  75(1)        \\
     20000 & 10 &   5 & 0.51(1) &  51(1)        \\
     10000 & 10 &  10 & 1.02(1) &  25.50(25)    \\
      9000 & 10 &  10 & 0.88(1) &  22.00(25)    \\
      7000 & 10 &  10 & 0.68(1) &  17.00(25)    \\
      5000 & 10 &  10 & 0.49(1) &  12.25(25)    \\
      3000 & 10 &  20 & 1.28(1) &   8.0000(625) \\
      2000 & 10 &  20 & 0.87(1) &   5.4375(625) \\
      1000 & 10 &  20 & 0.42(1) &   2.6250(625) \\
       900 & 10 &  50 & 2.02(1) &   2.02(1)     \\
       700 & 10 &  50 & 1.65(1) &   1.65(1)     \\
       500 & 10 &  50 & 1.23(1) &   1.23(1)     \\
       300 & 10 &  50 & 0.68(1) &   0.68(1)     \\
       200 & 10 & 100 & 1.98(1) &   0.4950(25)  \\
       100 & 10 & 100 & 0.98(1) &   0.2450(25)  \\
        90 & 10 & 100 & 1.08(1) &   0.2700(25)  \\
        70 & 10 & 100 & 0.95(1) &   0.2375(25)  \\
        30 & 10 & 100 & 1.20(1) &   0.3000(25)  \\
        20 & 10 &  50 & 0.80(1) &   0.80(1)     \\
        10 & 10 &  20 & 0.65(1) &   4.0625(625) \\
         5 & 10 &  20 & 1.30(1) &   8.1250(625) \\
    \bottomrule
  \end{tabular}
  \caption{Messdaten zur Untersuchung einer Reinmetallkathode, bei einer
  konstanten Vorverstärkung von 1000 und Gleichspannungsverstärkung von 10. Die
  Spannungen sind auf eine Nachverstärkung von 50 vereinheilicht.}
  \label{tab:reinmetallkathode}
\end{table}
