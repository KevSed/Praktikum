\subsection{Untersuchung des Stromrauschens einer Oxydkathode}

Im Folgenden wird das Stromrauschen einer Oxydkathode genauer untersucht. Die
Kathode wird mit einem Kathodenstrom von~$I=\SI{1}{\milli\ampere}$ betrieben,
was außerhalb des Sättigungsbereichs liegt. Die Spannungen, die bei gegebener
Mittenfrequenz des Band- bzw. Selektivverstärkers am Arbeitswiderstand der
Kathode abfallen, sind in Tabelle~\ref{tab:oxydkathode} gezeigt.

Gemäß Formel~\ref{...} in Abschnitt~\ref{...} berechnet sich die
Frequenzverteilung aus den mittleren Rauschspannungsquadraten zu
%
\begin{equation}
  W(\nu)=\frac{\bar{U}^2}{R^2\symup{\Delta}\nu}
\end{equation}
%
Die benötigten Bandbreiten~$\symup{\Delta}\nu$ bestimmen sich mit Hilfe einer
gegebenen empirisch ermittelten Tabelle, die dem Anhang beigefügt ist.
Abbildung~\ref{...} zeigt die Frequenzverteilung~$W$ aufgetragen gegen die
Mittenfrequenz~$\nu$. Der in Abschnitt~\ref{...} beschriebene Funkel-Effekt ist besonders im niederfrequenten Bereich des Spektrums dominant. Zur Bestimmung des Exponenten~$\alpha$ in der Formel für den Funkel-Effekt
%
\begin{equation}
  W(\nu)\prop\frac{I_0^2}{\nu^{\alpha}}
\end{equation}
%
wird für den linearen Anteil der Frequenzverteilung eine Ausgleichsrechnung durchgeführt. Der Exponent~$\alpha$ wird damit zu
%
\begin{equation}
  \alpha=\num{0(0)}
\end{equation}
%
bestimmt.
