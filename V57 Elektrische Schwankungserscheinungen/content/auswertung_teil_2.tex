\subsection{Untersuchung des Stromrauschens einer Oxydkathode}

Im Folgenden wird das Stromrauschen einer Oxydkathode genauer untersucht. Die
Kathode wird mit einem Kathodenstrom von~$I=\SI{1}{\milli\ampere}$ betrieben,
was außerhalb des Sättigungsbereichs liegt. Die Spannungen, die bei gegebener
Mittenfrequenz~$\nu$ des Band- bzw. Selektivverstärkers am Arbeitswiderstand~$R$
der Kathode abfallen, sind in Tabelle~\ref{tab:oxydkathode} gezeigt.

Gemäß Formel~\ref{...} in Abschnitt~\ref{sec:theorie} berechnet sich die
Frequenzverteilung aus den mittleren Rauschspannungsquadraten zu
%
\begin{equation}
  W(\nu)=\frac{\bar{U}^2}{R^2\symup{\Delta}\nu}
\end{equation}
%
Die benötigten Bandbreiten~$\symup{\Delta}\nu$ bestimmen sich mit Hilfe der
Werte für~$\nu$ aus einer gegebenen empirisch ermittelten Tabelle.
Abbildung~\ref{fig:oxydkathode_frequenzspektrum} zeigt die
Frequenzverteilung~$W$. Der in Abschnitt~\ref{sec:theorie} beschriebene
Funkel-Effekt ist besonders im niederfrequenten Bereich des Spektrums dominant.
Zur Bestimmung des Exponenten~$\alpha$ in der Formel für den Funkel-Effekt
%
\begin{equation}
  W(\nu)\propto\frac{I_0^2}{\nu^{\alpha}}
\end{equation}
%
wird für den linearen Anteil der Frequenzverteilung eine Ausgleichsrechnung
durchgeführt. Abbildung~\ref{fig:oxydkathode_frequenzspektrum_linearer_teil}
zeigt den linearen Teil der Frequenzverteilung, sowie die Ausgleichsgerade. Der
Exponent~$\alpha$ wird damit zu
%
\begin{equation}
  \alpha=\num{1.092(28)}
\end{equation}
%
bestimmt.

\begin{figure}
  \includegraphics[width=\textwidth]{analysis/oxydkathode_frequenzspektrum.pdf}
  \caption{Frequenzverteilung des Stromrauschens einer Oxydkathode. Zur besseren
  Übersichtlichkeit sind die Unsicherheiten der Messwerte nicht mit
  eingezeichnet.}
  \label{fig:oxydkathode_frequenzspektrum}
\end{figure}

\begin{figure}
  \includegraphics[width=\textwidth]{analysis/oxydkathode_frequenzspektrum_linearer_teil.pdf}
  \caption{Linearer Anteil der Frequenzverteilung des Stromrauschens einer
  Oxydkathode. Aus der Steigung der Ausgleichsgeraden wird der Exponent~$\alpha$
  in der Formel für den Funkel-Effekt bestimmt.}
  \label{fig:oxydkathode_frequenzspektrum_linearer_teil}
\end{figure}

\subsection{Untersuchung des Stromrauschens einer Reinmetallkathode}

Abschließend wird noch das Stromrauschen einer Reinmetallkathode untersucht.
Dazu werden zunächst die charakterisitischen Linien der Kathode aufgenommen, um
zu überprüfen, bei welcher Anodenspannung die Kathode im gesättigten Betrieb
arbeitet. Dieser Betrieb ist erreicht, wenn die Erhöhung der Anodenspannung
nicht mehr zu einer Erhöhung des Anodenstroms führt.
Abbildung~\ref{fig:reinmetallkathode_linien} zeigt den Anodenstrom aufgetragen
gegen die Anodenspannung bei festen Heizströmen. Ab einer Anodenspannung von
ungefähr~\SI{90}{\volt} arbeitet die Kathode im gesättigten Betrieb. Für die
folgenden Versuchsteile wird als Anodenspannung~\SI{120}{\volt} gewählt.

\begin{figure}
  \includegraphics[width=\textwidth]{analysis/reinmetallkathode_linien.pdf}
  \caption{Charakteristische Linien der Reinmetallkathode bei verschiedenen
  Heizströmen. Ab einer Anodenspannung von ungefähr~\SI{90}{\volt} arbeitet die
  Kathode im gesättigten Betrieb.}
  \label{fig:reinmetallkathode_linien}
\end{figure}

Tabelle~\ref{...} zeigt analog zu den vorherigen Versuchsteilen die
aufgenommenen Spannungen~$U$ bei gegebener Mittenfrequenz~$\nu$.
Abbildung~\ref{fig:reinmetallkathode_frequenzspektrum} zeigt die
Frequenzverteilung. Wie bei der Oxydkathode sind zwei Bereiche erkennbar. Für
niedrige Frequenzen ist die Verteilung linear abnehmend. In diesem Bereich
dominiert der Funkel-Effekt. Für größere Frequenzen ab~\SI{100}{\hertz} ist die
Verteilung annähernd konstant, da hier das frequenzunabhängige weiße Rauschen
maßgeblich ist.

\begin{figure}
  \includegraphics[width=\textwidth]{analysis/reinmetallkathode_frequenzspektrum.pdf}
  \caption{Frequenzverteilung des Stromrauschens einer Reinmetallkathode. Zur
  besseren Übersichtlichkeit sind die Unsicherheiten der Messwerte nicht mit
  eingezeichnet.}
  \label{fig:reinmetallkathode_frequenzspektrum}
\end{figure}

Zur Bestimmung der Elementarladung~$e_0$ wird nun das mittlere
Rauschstromquadrat~$\bar{I}^2$ gegen die Bandbreite~$\symup{\Delta}\nu$
aufgetragen, denn aus Gleichung~\eqref{...} wird ein linearer Zusammenhang
erwartet. Mit Hilfe der linearen Steigung kann dann auf die Elementarladung
geschlossen werden. Abbildung~\ref{...} zeigt die gemessenen Punkte. Da für
niedrige (NF) und hohe (HF) Frequenzen deutlich andere lineare Verläufe zu
erkennen sind, wird für die beiden Bereiche unabhängig voneinander eine
Ausgleichsrechnung durchgeführt. Es ergibt sich
%
\begin{align}
    \bar{I}^2_{\symup{NF}}&=\SI{0.0(0)}{\ampere\squared\second}+\SI{0.0(0)}{\ampere\squared} \\
    \bar{I}^2_{\symup{HF}}&=\SI{0.0(0)}{\ampere\squared\second}+\SI{0.0(0)}{\ampere\squared}
\end{align}
%
Die Steigungen~$m$ der Geraden sind gemäß Gleichung~\eqref{...} gegeben durch
%
\begin{equation}
  m=2e_0I_0
\end{equation}
%
Der anliegende Anodengleichstrom beträgt~$I_0=\SI{0}{\milli\ampere}$. Daraus
ergibt sich die Elementarladung zu
%
\begin{align}
  e_{0\symup{, NF}}&=\SI{1.594(10)e-19}{\coulomb} \\
  e_{0\symup{, HF}}&=\SI{1.19(9)e-19}{\coulomb}
\end{align}
%
Die Abweichung der beiden Werte...

\begin{figure}
  \includegraphics[width=\textwidth]{analysis/reinmetallkathode_elementarladung.pdf}
  \caption{Eine Caption.}
  \label{fig:reinmetallkathode_elementarladung}
\end{figure}
