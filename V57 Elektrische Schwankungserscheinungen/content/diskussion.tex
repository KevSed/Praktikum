\section{Diskussion}
\label{sec:diskussion}

\subsection{Ergebnisse aus der Vermessung der Einfachschaltung}

Die Kalibrationsmessung für die Einfachschaltung ergibt mit einem Wert von
$\SI{3500(70)}{\hertz}$ ein erwartet großes Frequenzband. \textbf{[Really?]}
Die Messungen zum Rauschspannungsquadrat der Einfachschaltung liefern wie
theoretisch erwartet einen linearen Zusammenhang, aus dem sich die
\textsc{Boltzmann}-Konstante bestimmen lässt. Die Ergebnisse für die beiden
Widerstände weichen dabei um $\SI{38}{\percent}$ (starker Widerstand), sowie
$\SI{52}{\percent}$ (schwacher Widerstand) vom Literaturwert
$\SI{1.38064e-23}{\joule\second}$\cite{CODATA} ab. Obwohl die Ermittelung des
eingeregelten Widerstandes am starken Widerstand deutlich größere Fluktuationen
aufwies, liegt der ermittelte Wert für $k_\text{B}$ näher am Literaturwert. Dass
diese Ergebnisse im Vergleich zur Korrelatorschaltung deutlich stärker abweichen
hängt sicherlich mit [...] zusammen.
Die verwendete Einfachschaltung besitzt eine Rauschzahl von
$F_\text{einf}=0,46\,\pm\,0,01$. Im Vergleich zur Korrelatorschaltung, welche
eine Rauschzahl von $F_\text{korr}=0,79\,\pm\,0,02$ besitzt, ist das
Eigenrauschen der Einfachschaltung also deutlich größer.

\subsection{Ergebnisse aus der Vermessung der Korrelatorschaltung}

Aus der Kalibrationsmessung der Korrelatorschaltung folgt im Vergleich zur
Einfachschaltung mit $\SI{900(20)}{\hertz}$ ein deutlich kleineres Frequenzband.
Aus dem thermischen Rauschen der beiden Widerstände folgen für die
\textsc{Boltzmann}-Konstante zwei Werte, welche sehr nah am Literaturwert
liegen. Mit einer Abweichung von $\SI{3}{\percent}$ (starker Widerstand), sowie
$\SI{26}{\percent}$ (schwacher Widerstand) liegen die ermittelten Werte deutlich
näher am Literaturwert, als die Ergebnisse der Einfachschaltung. Allerdings
liegt auch hier eine Diskrepanz zwischen den beiden Widerständen vor, wobei der
starke Widerstand wiederum besser abschneidet. Insgesamt liefert die
Korrelatorschaltung aufgrund des erwarteten geringeren Eigenrauschens exaktere
Ergebnisse, als die Einfachschaltung.

\subsection{Ergebnisse zum Stromrauschen zweier Kathoden}

Die Untersuchung der Oxydkathode liefert für den Exponenten~$\alpha$ in der
Formel für den Funkel-Effekt den Wert
%
\begin{equation}
  \alpha=\num{1.091(28)}
\end{equation}
%
Im theoretischen Modell zum Funkel-Effekt beträgt der Wert~$\alpha\approx 1$.
Mit einer Abweichung von~\SI{9.2}{\percent} nach oben vom exakten Wert 1 ist die
Messung konsistent mit der theoretischen Vorhersage.

Aus der Untersuchung der Reinmetallkathode werden zwei Werte für die
Elementarladung~$e_0$ bestimmt.
%
\begin{align}
  e_{0\symup{, NF}}&=\SI{1.594(10)e-19}{\coulomb} \\
  e_{0\symup{, HF}}&=\SI{1.19(9)e-19}{\coulomb}
\end{align}
%
Damit ergibt sich im niederfrequenten Bereich eine Abweichung vom
Literaturwert~\cite{CODATA}
%
\begin{equation}
  e_{0\symup{, lit}}=\SI{1.6021766208(98)e-19}{\coulomb}
\end{equation}
%
um circa~\SI{0.5}{\percent} nach unten. Für den hochfrequenten Bereich beträgt
die Abweichung rund~\SI{26}{\percent} nach unten. Somit liefert die Messung im
niederfrequenten Bereich im Gegensatz zur Messung im hochfrequenten Bereich ein
gutes Ergebnis.
%
\nocite{V57}
