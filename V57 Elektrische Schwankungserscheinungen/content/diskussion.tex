\section{Diskussion}
\label{sec:diskussion}

\subsection{Ergebnisse zum Stromrauschen zweier Kathoden}

Die Untersuchung der Oxydkathode liefert für den Exponenten~$\alpha$ in der
Formel für den Funkel-Effekt den Wert
%
\begin{equation}
  \alpha=\num{1.091(28)}
\end{equation}
%
Im theoretischen Modell zum Funkel-Effekt beträgt der Wert~$\alpha\approx 1$.
Mit einer Abweichung von~\SI{9.2}{\percent} nach oben vom exakten Wert 1 ist die
Messung konsistent mit der theoretischen Vorhersage.

Aus der Untersuchung der Reinmetallkathode werden zwei Werte für die
Elementarladung~$e_0$ bestimmt.
%
\begin{align}
  e_{0\symup{, NF}}&=\SI{1.594(10)e-19}{\coulomb} \\
  e_{0\symup{, HF}}&=\SI{1.19(9)e-19}{\coulomb}
\end{align}
%
Damit ergibt sich im niederfrequenten Bereich eine Abweichung vom
Literaturwert~\cite{CODATA}
%
\begin{equation}
  e_{0\symup{, lit}}=\SI{1.6021766208(98)e-19}{\coulomb}
\end{equation}
%
um circa~\SI{0.5}{\percent} nach unten. Für den hochfrequenten Bereich beträgt
die Abweichung rund~\SI{26}{\percent} nach unten. Somit liefert die Messung im
niederfrequenten Bereich im Gegensatz zur Messung im hochfrequenten Bereich ein
gutes Ergebnis.
%
\nocite{V57}
