\section{Diskussion}
\label{sec:diskussion}

Die Bestimmung des Kontrastes des Sagnac-Interferometers ergab in
dieser Versuchsdurchführung einen Verlauf der etwa durch eine
quadrierte sinus-Funktion beschrieben werden kann. Theoretisch
erwartet wird eine Zusammenhang der dem Produkt trigonometrischer
Funktionen entspricht. Daher ist der ermittelte Verlauf innerhalb
dieser Erwartung. Davon weicht allerdings die Tatsache ab, dass die
beiden lokalen Kotrastmaxima unterschiedlich hoch sind. Real hat dies
allerdings lediglich zur Folge, dass bei $\varphi=\SI{45}{\degree}$
und einem Kontrast von $K=0.8$ gemessen wird.

Die Untersuchung der Glasplatte ergab einen Brechungsindex von
%
\begin{equation*}
  n_\text{Glas}=1,24\,\pm\,0.26 \, .
\end{equation*}
%
Dieser Wert ist mit einem relativen Fehler von rund
$\SI{80}{\percent}$ mit einer großen Unsicherheit behaftet.
Allerdings zeigt die Abweichung des Nominalwertes vom erwarteten
Brechungsindex von Glas ($n_\text{Literatur}\approx
1,5$~\cite{??}) mit etwa $\SI{17}{\percent}$ eine gute Genauigkeit
dieses Wertes. Daher ist davon auszugehen, dass dieser Messteil durch
bessere Statistik noch weiter verbessert werden kann.

Die beiden untersuchten Gase $\symup{CO}_2$ und Luft zeigten in der versuchsdurchführung Brechungsindices, die unter Normalbedingungen
%
\begin{align*}
  n_\text{Luft}(p=\,1\,\text{atm})=& 1,000\,271\,8\,\pm\,0,000\,000\,5   \;\text{und} \\
  n_{\symup{CO}_2}(p=\,1\,\text{atm})=& 1,000\,397\,\pm\,0,000\,006
\end{align*}
%
betragen. [Literaturwerte]
Die Übereinstimmung mit den Literaturwerten ist dabei sehr gut. In Anbetracht der nicht allzu extensiven Wertenahme liefert das Interferometer dennoch gute Messwerte. Systematische Fehlerquellen sind etwa bei der Dichtigkeit der verwendeten Kammer und somit einer Vermischung des $\symup{CO}_2$ mit Luft zu suchen. Dies würde den bestimmten Wert des Brechungsindex herabsenken; der brechungsindex von Luft liegt nämlich unter dem von $\symup{CO}_2$. Auch die Messinstrumente bergen systematische Fehlerquellen.
