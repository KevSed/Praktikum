\section{Diskussion}
\label{sec:diskussion}

Die Bestimmung des Kontrastes des Sagnac-Interferometers ergab in
im Versuch einen Verlauf der gut durch die Theoriekurve~\eqref{eq:kontrast2}
beschrieben wird. Für alle weiteren Versuchsteile wurde bei maximalen Kontrast
gemessen.

Die Untersuchung des Glasplättchens ergab einen Brechungsindex von
\begin{equation}
  n_{\text{Glas}}=\num{1.54(2)}
\end{equation}
Dieser Wert ist mit einem relativen Fehler von rund $\SI{1.2}{\percent}$ sehr
präzise. Der Nominalwert selbst liegt im Bereich um
$n_\text{Literatur}\approx 1,5$, der für Glas erwartet wird. Es ist davon
auszugehen, dass der Messteil sehr gute Ergebnisse geliefert hat.

Die beiden untersuchten Gase $\text{CO}_2$ und Luft zeigten in der
Versuchsdurchführung Brechungsindices, die unter Normalbedingungen
%
\begin{align}
  n_{\text{Luft}}(p=1\,\text{atm})=&\num{1.000280(1)} \\
  n_{\text{CO}_2}(p=1\,\text{atm})=&\num{1.000409(6)}
\end{align}
%
betragen. Die Übereinstimmung mit den Literaturwerten~\cite{brechungsindices}
%
\begin{align}
  n_{\text{Luft}\text{,Lit}}(p=1\,\text{atm})=&\num{1.000292} \\
  n_{\text{CO}_2\text{,Lit}}(p=1\,\text{atm})=&\num{1.000449}
\end{align}
%
ist recht gut. Der gemessene Wert von Luft weicht um
rund~$\SI{4}{\percent}$ vom Literaturwert ab, der von $\text{CO}_2$ um rund
$\SI{9}{\percent}$, wobei zur Bestimmung der Abweichungen die Formel
%
\begin{equation}
  \frac{(n_{\text{Messung}}-1)-(n_{\text{Literatur}}-1)}
  {(n_{\text{Literatur}}-1)}
\end{equation}
%
verwendet wurde. In Anbetracht der nicht allzu extensiven Wertenahme liefert das
Interferometer durchaus gute Messwerte. Systematische Fehlerquellen sind etwa bei
der Dichtigkeit der verwendeten Kammer und somit einer Vermischung des
$\text{CO}_2$ mit Luft zu suchen. Dies würde den bestimmten Wert des
Brechungsindex herabsenken; der Brechungsindex von Luft liegt nämlich unter dem
von $\symup{CO}_2$. Auch die Messinstrumente bergen systematische Fehlerquellen.
