\section{Auswertung}
\label{sec:auswertung}

Die in der Auswertung verwendeten Mittelwerte mehrfach gemessener Größen sind gemäß der Gleichung
%
\begin{equation}
    \bar{x}=\frac{1}{n}\sum_{i=1}^n x_i
    \label{eq:mittelwert}
\end{equation}
%
bestimmt.
Die Standardabweichung des Mittelwertes ergibt sich dabei zu
%
\begin{equation}
    \symup{\Delta}\bar{x}=\sqrt{\frac{1}{n(n-1)}\sum_{i=1}^n\left(x_i-\bar{x}\right)^2}.
    \label{eq:standardabweichung}
\end{equation}
%
Resultiert eine Größe über eine Gleichung aus zwei anderen fehlerbehafteten Größen, so berechnet sich der Gesamtfehler nach der Gaußschen Fehlerfortpflanzung zu
%
\begin{equation}
    \symup{\Delta}f(x_1,x_2,...,x_n)=\sqrt{\left(\frac{\partial f}{\partial x_1}\symup{\Delta}x_1\right)^2+\left(\frac{\partial f}{\partial x_2}\symup{\Delta}x_2\right)^2+ \dotsb +\left(\frac{\partial f}{\partial x_n}\symup{\Delta}x_n\right)^2}.
    \label{eq:fehlerfortpflanzung}
\end{equation}
%
Alle in der Auswertung angegebenen Größen sind stets auf die erste signifikante Stelle des Fehlers gerundet.
Setzt sich eine Größe über mehrere Schritte aus anderen Größen zusammen, so wird erst am Ende gerundet, um Fehler zu vermeiden.
Zur Auswertung wird die Programmiersprache \texttt{python (Version 3.4.1)}
mit den Bibliothekserweiterungen \texttt{numpy}, \texttt{scipy} und \texttt{matplotlib} zur Erstellung der Grafiken und linearen Regressionen verwendet.


\subsection{Kontrastbestimmung}
Zu Beginn wird das Kontrastmaximum des Sagnac-Interferometers gesucht, d.h. die
Einstellung, bei welcher das erzeugte Interferenzbild seine maximale Schärfe
besitzt. Dazu werden die maximale und die minimale Stromstärke an der
Photodiode für verschiedene Winkel des Polarisationsfilters über Variation der
Ausrichtung des Glasplättchens gemessen. Der Kontrast $K$ lässt sich
über den Zusammenhang~\eqref{eq:kontrast1} berechnen. Er folgt in Abhängigkeit
des Winkels $\varphi$ einem Verlauf der Gleichung~\eqref{eq:kontrast2}
entspricht. Eine Ausgleichsrechnung mit den Messdaten liefert für die Parameter
\begin{align}
  A&=\num{0.753(9)} \\
  \delta&=\num{-0.12(1)}
\end{align}
Die Messdaten sind in Tabelle~\ref{tab:kontrast} aufgeführt. Der angepasste
Verlauf ist zusammen mit den Messdaten in Abbildung~\ref{fig:kontrast}
dargestellt.

\begin{table}
  \centering
  \begin{tabular}{S[table-format=3.0]
                  S[table-format=3.0]
                  S[table-format=4.0]
                  S[table-format=1.3]}
    \toprule
    {$\varphi/\si{\degree}$} & {$U_{\text{min}}/\si{\milli\volt}$} &
    {$U_{\text{max}}/\si{\milli\volt}$} & {Kontrast} \\
    \midrule
      0 & 458 &  530 & 0.072 \\
     15 & 400 &  740 & 0.298 \\
     30 & 270 &  910 & 0.542 \\
     35 & 206 & 1008 & 0.799 \\
     40 & 167 & 1022 & 0.722 \\
     45 & 103 &  923 & 0.490 \\
     50 & 121 &  848 & 0.125 \\
     55 & 116 &  827 & 0.303 \\
     60 & 118 &  732 & 0.637 \\
     75 & 162 &  474 & 0.745 \\
     90 & 192 &  247 & 0.651 \\
    105 & 115 &  215 & 0.407 \\
    120 &  54 &  244 & 0.081 \\
    125 &  47 &  249 & 0.660 \\
    130 &  46 &  302 & 0.719 \\
    135 &  47 &  322 & 0.750 \\
    140 &  55 &  373 & 0.753 \\
    145 &  76 &  412 & 0.682 \\
    150 &  97 &  459 & 0.735 \\
    165 & 249 &  591 & 0.742 \\
    180 & 489 &  576 & 0.688 \\
    \bottomrule
  \end{tabular}
  \caption{Messwerte und berechnete Kontraste.}
  \label{tab:kontrast}
\end{table}

\begin{figure}[htb]
  \centering
  \includegraphics[width=0.8\textwidth]{analysis/kontrast.pdf}
  \caption{Messwerte für den winkelabhängigen Verlauf des Kontrastes des
  verwendeten Interferometers, sowie die in \eqref{eq:kontrast2} beschriebene
  Funktion.}
  \label{fig:kontrast}
\end{figure}

Hohe Kontraste ergeben sich um die Winkel $\varphi_1=\SI{45}{\degree}$ und
$\varphi_2=\SI{135}{\degree}$. Das Interferometer wird im Weiteren so
eingestellt, dass bei hohem Kontrast gemessen wird.

\subsection{Bestimmung des Brechungsindex von Glas}
Der Brechungsindex von Glas wird über die Zählung der Interferenzmaxima bei
gleichmäßiger Rotation des Doppelglashalters um~$\SI{10}{\degree}$ bestimmt.
Dazu werden zehn unabhängige Messungen durchgeführt. Die im Versuch verwendeten
Glasplättchen haben eine Dicke von~$T=\SI{1}{\milli\meter}$ und der Laser eine
Wellenlänge von~$\lambda=\SI{632.8}{\nano\meter}$. Im Mittel
werden~$\num{34.4(10)}$ Interferenzmaxima gezählt. Mit Hilfe von
Gleichung~\eqref{eq:glasplättchen} und der Bedingung~$M(\theta=0)=0$ wird mit
einer Ausgleichsrechnung der Brechungsindex von Glas zu
\begin{equation}
  n_{\text{Glas}}=\num{1.54(2)}
  \label{eq:glas}
\end{equation}
bestimmt.

\subsection{Bestimmung der Brechungsindices von Kohlenstoffdioxid und Luft}
Nun wird das Lichtbrechungsverhalten zweier Gase in Abhängigkeit des Gasdrucks
untersucht. Dazu werden das Gas Kohlenstoffdioxid sowie das Gasgemisch Luft
gewählt. In einer lichtdurchlässigen Kammer lässt sich der Gasdruck mit Hilfe
einer Pumpe regulieren, sodass der Phasenunterschied reguliert und somit die
Brechungsindices der Gase über eine Zählung von Interferenzmaxima bestimmt
werden können. Die Länge der in diesem Experiment verwendeten Kammer beträgt
$L=\SI{100.0(1)}{\milli\meter}$~\cite{V64}. Die Messung wird über einen
Druckbereich von etwa $\SI{50}{\milli\bar}$ bis $\SI{1000}{\milli\bar}$ jeweils
dreimal durchgeführt. Gleichung~\eqref{eq:brechungsindex_gas} liefert einen
Zusammenhang zwischen der gemessenen Anzahl von Interferenzmaxima und dem
Brechungsindex. In den Abbildungen~\ref{fig:luft1} - \ref{fig:co2_3} sind
die bestimmten Brechungsindices in Abhängigkeit des Gasdrucks, sowie lineare
Ausgleichsgeraden der Form
\begin{equation}
  n(p)=m\cdot p+b
  \label{eq:linear}
\end{equation}
aufgetragen, wobei die Ausgleichsfunktion durch Gleichung~\eqref{eq:näherung}
motiviert ist. Die Ergebnisse der Ausgleichsrechnungen sind in
Tabelle~\ref{tab:params} zusammengetragen.

\begin{table}
  \centering
  \begin{tabular}{lll}
    \toprule
    {\textbf{Luft}} & {a/$\si{\milli\bar^{-1}}$} & {$b-1$} \\
    \midrule
    Messung 1 & \num{266(2)e-9} & \num{ 5(10)e-7} \\
    Messung 2 & \num{268(2)e-9} & \num{-3(10)e-7} \\
    Messung 3 & \num{268(2)e-9} & \num{-6(13)e-7} \\
    \midrule
    {\textbf{CO}$_2$} & {a/$\si{\milli\bar^{-1}}$} & {$b-1$} \\
    \midrule
    Messung 1 & \num{396(2)e-9} & \num{-16(1)e-6} \\
    Messung 2 & \num{400(2)e-9} & \num{ -3(1)e-6} \\
    Messung 3 & \num{400(2)e-9} & \num{ -5(1)e-6} \\
    \bottomrule
  \end{tabular}
  \caption{Ergebnisse für die Parameter der linearen Ausgleichsrechnungen für
           die drei Messungen der jeweiligen Gase.}
  \label{tab:params}
\end{table}

Aus den Messreihen ergeben sich für die linearen Zusammenhänge
\begin{align}
  n_{\text{Luft}}(p)=&\SI{2672(8)e-10}{\milli\bar^{-1}}
  \cdot p+1-\num{2(5)e-7} \\
  n_{\text{CO}_2}(p)=&\SI{398(2)e-9}{\milli\bar^{-1}}
  \cdot p+1-\num{8(6)e-6}
\end{align}
Mit Hilfe der bestimmten Parameter und des zur Versuchsdurchführung herrschenden
Umgebungsdrucks und der Umgebungstemperatur lassen sich die Brechungsindices der
Gase berechnen. Die nächstgelegene Wetterstation hat zur Versuchsdurchführung
einen Luftdruck von $\SI{1018}{\hecto\pascal}$~\cite{wetteronline} verzeichnet
Daraus folgt für die Brechungsindices
\begin{align}
  n_{\text{Luft}}(p=\SI{1018}{\milli\bar})=&\num{1.0001359(5)} \\
  n_{\text{CO}_2}(p=\SI{1018}{\milli\bar})=&\num{1.0001986(30)}
\end{align}
Nach Gleichung~\eqref{eq:näherung} ist der Proportionalitätsfaktor
der Druckabhängigkeit für Gase mit einem Brechungsindex nahe $1$ etwa
$\frac{3A}{RT}$. Es gilt also
\begin{equation}
  m=\frac{3A}{RT}
\end{equation}
wobei $m$ die aus der Ausgleichsrechnung bestimmte Steigung, $A$ die molare
Refraktivität des Mediums, $R$ die universelle Gaskonstante und $T$ die
Temperatur ist. Somit lässt sich der aus den Messdaten bestimmte Wert für $m$
auf Normalbedingungen umrechnen. Zur Berechnung der Brechungsindices unter
Normalbedingungen ($T_0=\SI{15}{\celsius}$, $p=1\,\text{atm}$) gilt
\begin{equation}
  m_0=\frac{mT}{T_0}
\end{equation}
Daraus folgt
\begin{align}
  n_{\text{Luft}}(p=1\,\text{atm})=&\num{1.000280(1)} \\
  n_{\text{CO}_2}(p=1\,\text{atm})=&\num{1.000409(6)}
\end{align}

\begin{figure}[htb]
  \centering
  \includegraphics[width=0.7\textwidth]{analysis/Luft_1.pdf}
  \caption{Erste Messung des Brechungsindex von Luft. Aufgetragen sind die aus
  den Messwerten bestimmten Brechungsindices, sowie die Ausgleichsgerade. Die
  Fehler der Messwerte wurden um den Faktor 100 vergrößert.}
  \label{fig:luft1}
\end{figure}

\begin{figure}
  \centering
  \includegraphics[width=0.7\textwidth]{analysis/Luft_2.pdf}
  \caption{Zweite Messung des Brechungsindex von Luft. Aufgetragen sind die aus
  den Messwerten bestimmten Brechungsindices, sowie die Ausgleichsgerade. Die
  Fehler der Messwerte wurden um den Faktor 100 vergrößert.}
  \label{fig:luft2}
\end{figure}

\begin{figure}
  \centering
  \includegraphics[width=0.7\textwidth]{analysis/Luft_3.pdf}
  \caption{Dritte Messung des Brechungsindex von Luft. Aufgetragen sind die aus
  den Messwerten bestimmten Brechungsindices, sowie die Ausgleichsgerade. Die
  Fehler der Messwerte wurden um den Faktor 100 vergrößert.}
  \label{fig:luft3}
\end{figure}%

\begin{figure}
  \centering
  \includegraphics[width=0.7\linewidth]{analysis/CO2_1.pdf}
  \caption{Erste Messung des Brechungsindex von Kohlenstoffdioxid. Aufgetragen
  sind die aus den Messwerten bestimmten Brechungsindices, sowie die
  Ausgleichsgerade. Die Fehler der Messwerte wurden um den Faktor 100
  vergrößert.}
  \label{fig:co2_1}
\end{figure}

\begin{figure}
  \centering
  \includegraphics[width=0.7\linewidth]{analysis/CO2_2.pdf}
  \caption{Zweite Messung des Brechungsindex von Kohlenstoffdioxid. Aufgetragen
  sind die aus den Messwerten bestimmten Brechungsindices, sowie die
  Ausgleichsgerade. Die Fehler der Messwerte wurden um den Faktor 100
  vergrößert.}
  \label{fig:co2_2}
\end{figure}

\begin{figure}
  \centering
  \includegraphics[width=0.7\linewidth]{analysis/CO2_3.pdf}
  \caption{Dritte Messung des Brechungsindex von Kohlenstoffdioxid. Aufgetragen
  sind die aus den Messwerten bestimmten Brechungsindices, sowie die
  Ausgleichsgerade. Die Fehler der Messwerte wurden um den Faktor 100
  vergrößert.}
  \label{fig:co2_3}
\end{figure}
