\section{Durchführung}
\label{sec:durchführung}

\subsection{Der Versuchsaufbau}

Abbildung~\ref{fig:aufbau} zeigt den schematischen Aufbau der Versuchsapparatur.
Ausgangspunkt des untersuchten Lichtes ist eine Spektrallampe. Dabei handelt es
sich um eine Rubidiumdampflampe. Das Licht dieser Lampe wird über eine
Sammellinse fokussiert und parallelisiert. Ein D1-Filter selektiert aus diesem
Licht ein schmales Frequenzspektrum heraus. Dabei handelt es sich um die
D1-Linie des Rubidiums, welche eine Wellenlänge von etwa
$\lambda=\SI{794.8}{\nano\metre}$ besitzt. Der Filter stellt ein von einer
refĺektierenden Schicht umgebenes Dielektrikum der Dicke $d$ und
Brechungsindex $n$ dar. Die äußere Schicht ist für Licht halbdurchlässig,
weswegen einfallendes Licht innerhalb des Dielektrikums mehrfach reflektiert
wird. Dies führt dazu, dass Wellenlängen der Größe
%
\begin{equation}
  m\cdot\lambda_m=2\cdot n\cdot d+\frac{\lambda}{2}, \quad m=2,3,\ldots
\end{equation}
%
konstruktiv interferieren. Licht einer anderen Wellenlänge hingegen wird so
durch destruktive Interferenz herausgefiltert. Durch einen Polarisationsfilter
wird das Licht anschließend linear polarisiert, um anschließend durch ein
$\sfrac{\lambda}{4}$-Plättchen zirkular polarisiert zu werden. Das
$\sfrac{\lambda}{4}$-Plättchen stellt eine aus einem anisotropen Material
besthende Verzögerungsvorrichtung dar. Sie ist in der Lage einfallendes Licht
entlang zweier Achsen unterschiedlich zu verzögern, sodass durch die Ausrichtung
der gewünschte Phasenunterschied zwischen einzelnen Komponenten des Lichtes
erzeugt werden kann. Für zirkular polarisiertes Licht aus linear polarisiertem Licht,
muss diese Ausrichtung gerade einen Winkel von $\sfrac{\pi}{2}$ betragen. \\ Das
nun hinreichend präparierte Licht fällt anschließend auf eine Dampfzelle. Diese
ist mit Rubidiumgas gefüllt und elektrisch beheizbar. Die Temperaturerhöhung
sorgt dafür, dass die thermische Verteilung der Energie breiter ist, sowie zur
Regulierung des Dampfdruckes innerhalb der Zelle. Die Zelle befindet sich
innerhalb mehrerer Helmholtzspulenpaare. Ein Paar ist horizontal ausgerichtet
(Horizontalfeldspule), während das zweite Paar vertikal ausgerichtet ist. Dazu
geschaltet ist eine Modulationsfeldspule (Sweep-Spule). Alle Spulen lassen sich
einzeln über die Feldströme steuern. Nachdem das Licht die Dampfzellle passiert
hat wird es durch eine Sammellinse wieder fokussiert und anschließend von einem
Lichtdetektor vermessen. Dazu wird ein Si-Photoelement verwendet, welches das
Licht in ein elektrisches Signal wandelt. Dieses wird durch ein Oszilloskop
eingelesen und dargestellt. Zur Abschirmung von Außenlicht wird der gesamte
Messaufbau während der Messung mit einer schwarzen Stoffdecke abgedunkelt.
Dennoch lässt sich die Messung durch Lichtschwankungen leicht beeinflussen.

\subsection{Die Versuchsdurchführung}

Da es sich hierbei um einen optischen Aufbau mit mehreren optischen Elementen
handelt, muss die Apperatur zunächst so eingestellt werden, dass das vermessene
Licht eine maximale Intensität aufweist. Dazu wird der Polarisationsfilter
sowie das $\sfrac{\lambda}{4}$-Plättchen zunächst aus dem Strahlengang entfernt.
Da die Magnetfeldstärken der verwendeten Spulen in der Größenordnung des
Erdmagnetfeldes liegen, muss dieses möglichst kompensiert werden. Dazu wird der
gesamte Versuchsaufbau entlang der optischen Achse in Nord-Süd-Richtung
ausgerichtet. Dies dient zur Kompensation der horizontalen Komponente des
Magnetfeldes. Die vertikale Komponente wird durch die Vertikalspule
ausgeglichen. Dazu wird der Spulenstrom solange variiert, bis der auf dem
Oszilloskop dargestellte Strom des Lichtdetektors eine minimale Breite aufweist.
Zur Verringerung des Einflusses äußeren Lichtes wird der Aufbau nun durch eine
schwarze Stoffdecke abgeschirmt.\\ Zur Vermessung der Zeeman-Aufspaltung wird an
den Aufbau ein äußeres niederfrequentes Magnetfeld angelegt. Dieses wird über
eine Sägezahnspannung an der Sweep-Spule erzeugt. Ziel ist es, die in
Abbildung~\ref{fig:transparenz_b} dargestellten \enquote{dips} auf dem
Oszilloskop zu untersuchen. Dazu ist es notwendig die Darstellung für
verschiedene Frequenzen des Sweep-Spulen-Stromes auf dem Oszilloskop anzupassen
und die Zeitdifferenz dieser \enquote{dips} abzulesen.\\ Anschließend wird das
oben beschriebene Überschwingen des Lichtdetektor-Stromes (also der Transparenz)
bei Anlegen einer Rechteckspannung an die Sweep-Spule untersucht. Dazu wird eine
Rechteckspannung variierender Frequenzen an die Sweep-Spule gelegt und an der
abfallenden Flanke dieser die gedämpfte Schwingung betrachtet. Durch Ausmessen
des Zeitabstandes mehrerer Maxima dieser Schwingung lässt sich die Frequenz
bestimmen.
