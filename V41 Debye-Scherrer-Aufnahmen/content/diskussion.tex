\section{Diskussion}
\label{sec:diskussion}

Zusammengefasst lässt die Untersuchung der Probe 8 auf eine Diamant-Struktur
mit einer Gitterkonstanten von~$\SI{5.5(2)}{\angstrom}$ schließen. Es könnte
sich somit um Zinksulfid handeln, welches in einer Diamantstruktur
kristallisiert und eine Gitterkonstante von~$\SI{5.41}{\angstrom}$~\cite{zinksulfid} besitzt.

Für das Salz 2 wurde die Gitterkonstante zu~$\SI{4.11(3)}{\angstrom}$ bestimmt. Da Cäsiumchlorid die
Gitterkonstante~$\SI{4.13}{\angstrom}$~\cite{cäsiumchlorid} besitzt, handelt es sich bei Salz 2
wahrscheinlich um Cäsiumchlorid.

Typische Fehlerquellen des Versuchs sind eine unsaubere Präparation der
bestrahlten Probe, sowie Fehler bei der Entwicklung und Auswerung der
Fotostreifen. In dem hier durchgeführten Versuch ist der zur Salzprobe
zugehörige Fotofilm stark geschwärzt worden. Eine Auswertung des Films konnte
nur unter starkem Gegenlicht erfolgen.
