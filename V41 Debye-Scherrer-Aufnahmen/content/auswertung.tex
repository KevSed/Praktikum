\section{Auswertung}
\label{sec:auswertung}

\subsection{Fehlerrechnung}

Die in der Auswertung verwendeten Mittelwerte mehrfach gemessener Größen sind gemäß der Gleichung
%
\begin{equation}
    \bar{x}=\frac{1}{n}\sum_{i=1}^n x_i
    \label{eq:mittelwert}
\end{equation}
%
bestimmt.
Die Standardabweichung des Mittelwertes ergibt sich dabei zu
%
\begin{equation}
    \symup{\Delta}\bar{x}=\sqrt{\frac{1}{n(n-1)}\sum_{i=1}^n\left(x_i-\bar{x}\right)^2}.
    \label{eq:standardabweichung}
\end{equation}
%
Resultiert eine Größe über eine Gleichung aus zwei anderen fehlerbehafteten Größen, so berechnet sich der Gesamtfehler nach der Gaußschen Fehlerfortpflanzung zu
%
\begin{equation}
    \symup{\Delta}f(x_1,x_2,...,x_n)=\sqrt{\left(\frac{\partial f}{\partial x_1}\symup{\Delta}x_1\right)^2+\left(\frac{\partial f}{\partial x_2}\symup{\Delta}x_2\right)^2+ \dotsb +\left(\frac{\partial f}{\partial x_n}\symup{\Delta}x_n\right)^2}.
    \label{eq:fehlerfortpflanzung}
\end{equation}
%
Alle in der Auswertung angegebenen Größen sind stets auf die erste signifikante Stelle des Fehlers gerundet.
Setzt sich eine Größe über mehrere Schritte aus anderen Größen zusammen, so wird erst am Ende gerundet, um Fehler zu vermeiden.
Zur Auswertung wird die Programmiersprache \texttt{python (Version 3.4.1)}
mit den Bibliothekserweiterungen \texttt{numpy}, \texttt{scipy} und \texttt{matplotlib} zur Erstellung der Grafiken und linearen Regressionen verwendet.


\subsection{Vorgehensweise}

In einem ersten Schritt werden die Netzebenen verschiedener elementarer Gitter
bestimmt, deren Reflexe nicht verschwinden. Dazu wird die Gleichung für die
Strukturamplitude~\eqref{eq:strukturamplitude} zugrunde gelegt. Durch Variation
der Indizes~$(hkl)$ werden die Kombinationen ausgemacht, für die die
Strukturamplitude nicht Null wird. In einem zweiten Schritt wird durch Einsetzen
von Gleichung~\eqref{eq:d} in Gleichung~\eqref{eq:bragg} der Ausdruck
%
\begin{equation}
  \frac{\lambda}{2\sin\theta}\sqrt{h^2+k^2+l^2}=d\sqrt{N}=a
  \label{eq:a}
\end{equation}
%
für die Gitterkonstante aufgestellt. Normieren liefert den Zusammenhang
%
\begin{equation}
  \frac{d_1}{d_i}=\sqrt{\frac{N_i}{N_1}}.
\end{equation}
%
Nun werden die normierten Wurzeln für die zulässigen Werte von~$N$ berechnet.
Tabelle~\ref{tab:theorie} listet die verschiedenen Werte für die verschiedenen
Gittertypen auf.

\begin{figure}[h]
  \centering
  \caption{Miller-Indizes der Netzebenen mit nicht verschwindenden
  Beugungsreflexen, sowie der Quotient $\sqrt{N_i/N_1}$ der jeweiligen Ebenen.}
  \begin{subtable}{0.45\textwidth}
    \centering
    \caption{sc-Gitter, Cäsiumchlorid-Struktur}
    \begin{tabular}{S[table-format=3]
                    S[table-format=2]
                    S[table-format=1.3]}
      \toprule
      {$hkl$}  & {$N$} & {$\sqrt{N_i/N_1}$} \\
      \midrule
      100 &  1 & 1.000 \\
      110 &  2 & 1.414 \\
      111 &  3 & 1.732 \\
      200 &  4 & 2.000 \\
      210 &  5 & 2.236 \\
      211 &  6 & 2.449 \\
      220 &  8 & 2.828 \\
      221 &  9 & 3.000 \\
      300 &  9 & 3.000 \\
      310 & 10 & 3.162 \\
      311 & 11 & 3.317 \\
      222 & 12 & 3.464 \\
      320 & 13 & 3.606 \\
      321 & 14 & 3.742 \\
      400 & 16 & 4.000 \\
      \bottomrule
    \end{tabular}
  \end{subtable}
  \begin{subtable}{0.45\textwidth}
    \centering
    \caption{bcc-Gitter}
    \begin{tabular}{S[table-format=3]
                    S[table-format=2]
                    S[table-format=1.3]}
      \toprule
      {$hkl$}  & {$N$} & {$\sqrt{N_i/N_1}$} \\
      \midrule
      110 &  2 & 1.000 \\
      200 &  4 & 1.414 \\
      211 &  6 & 1.732 \\
      220 &  8 & 2.000 \\
      310 & 10 & 2.236 \\
      222 & 12 & 2.449 \\
      321 & 14 & 2.646 \\
      400 & 16 & 2.828 \\
      330 & 18 & 3.000 \\
      411 & 18 & 3.000 \\
      420 & 20 & 3.162 \\
      332 & 22 & 3.317 \\
      422 & 24 & 3.464 \\
      431 & 26 & 3.606 \\
      510 & 26 & 3.606 \\
      \bottomrule
    \end{tabular}
  \end{subtable} \\ \vspace{1cm}
  \begin{subtable}{0.45\textwidth}
    \centering
    \caption{fcc-Gitter, Fluorit- und Steinsalz-Struktur}
    \begin{tabular}{S[table-format=3]
                    S[table-format=2]
                    S[table-format=1.3]}
      \toprule
      {$hkl$}  & {$N$} & {$\sqrt{N_i/N_1}$} \\
      \midrule
      111 &  3 & 1.000 \\
      200 &  4 & 1.155 \\
      220 &  8 & 1.633 \\
      311 & 11 & 1.915 \\
      222 & 12 & 2.000 \\
      400 & 16 & 2.309 \\
      331 & 19 & 2.517 \\
      420 & 20 & 2.582 \\
      422 & 24 & 2.828 \\
      333 & 27 & 3.000 \\
      511 & 27 & 3.000 \\
      440 & 32 & 3.266 \\
      531 & 35 & 3.416 \\
      442 & 36 & 3.464 \\
      600 & 36 & 3.464 \\
      \bottomrule
    \end{tabular}
  \end{subtable}
  \begin{subtable}{0.45\textwidth}
    \centering
    \caption{Diamant-Gitter, Zinkblende-Struktur}
    \begin{tabular}{S[table-format=3]
                    S[table-format=2]
                    S[table-format=1.3]}
      \toprule
      {$hkl$}  & {$N$} & {$\sqrt{N_i/N_1}$} \\
      \midrule
      111 &  3 & 1.000 \\
      220 &  8 & 1.633 \\
      311 & 11 & 1.915 \\
      400 & 16 & 2.309 \\
      331 & 19 & 2.517 \\
      422 & 24 & 2.828 \\
      333 & 27 & 3.000 \\
      511 & 27 & 3.000 \\
      440 & 32 & 3.266 \\
      531 & 35 & 3.416 \\
      620 & 40 & 3.651 \\
      533 & 43 & 3.786 \\
      444 & 48 & 4.000 \\
      551 & 51 & 4.123 \\
      711 & 51 & 4.123 \\
      \bottomrule
    \end{tabular}
  \end{subtable}
  \label{tab:theorie}
\end{figure}

Es wird nun versucht an Hand der experimentell angefertigten
Debye-Scherrer-Aufnahmen die Quotienten~$d_1/d_i$ zu bestimmen. Durch Vergleich
mit Tabelle~\ref{tab:theorie} kann dann auf den zugrundeliegenden Gittertypen
geschlossen werden.

Es erweist sich als vorteilhaft, dass der Umfang des
Kameragehäuses~$\SI{360}{\milli\metre}$ beträgt, da die gemessenen Abstände der
aufgenommenen Linien direkt als Winkelmaß~$2\theta$ in Grad übersetzt werden
können. Die Messung der Linienabstände auf dem entwickelten Fotofilm erfolgt mit
Hilfe eines Lineals. Als Ablesefehler wird~\SI{1}{\milli\metre} angenommen. Die
Austrittsöffnung der Kamera, an der die Röntgenstrahlung gestoppt wird, ist auf
dem Film mit einem Strich markiert.
Die Berechnung der Netzebenenabstände~$d$ erfolgt mit Hilfe der Bragg-Bedingung in
Gleichung~\eqref{eq:bragg}. Dabei ist die Wellenlänge der Röntgenstrahlung
mit~$\lambda=\SI{1.54286}{\angstrom}$ gegeben. Der Wert ergibt sich durch
Mittelung der Wellenlänge für die~$K_{\alpha,1}$- und die~$K_{\alpha,2}$-Linie der
verwendeten Röntgenquelle mit Kupferanode.

Nach erfolgter Bestimmung des Gittertyps werden in einem weiteren
Auswertungsschritt die Gitterkonstanten der untersuchten Proben bestimmt. Dazu
werden mit Hilfe von Gleichung~\eqref{eq:a} die Gitterkonstanten an Hand der
verschiedenen Messpunkte bestimmt. Durch einen linearen Fit der Messwerte
gegen~$\cos^2\theta$ kann dann der beste Wert für die Gitterkonstante als
$y$-Achsenabschnitt bestimmt werden.

\subsection{Untersuchung der Probe 8}

Die Untersuchung der Probe 8 liefert die in Tabelle~\ref{tab:probe8}
aufgeführten Ergebnisse. Dabei werden für kleine Radien, also kleine Beugungswinkel,
Doppelringe beobachtet, die durch das Vorhandensein der zwei unterschiedlichen
Wellenlängen erklärt werden können. Die Doppelringe für sehr große Beugungswinkel
sind Resultat einer unbekannten Störung. Im Folgenden werden stets die Mittelwerte
der beiden gemessenen Radien betrachtet. Die beobachteten Beugungsreflexe können am besten durch
ein Diamant-Gitter oder eine Zinkblende-Struktur erklärt werden.
Abbildung~\ref{fig:probe8} zeigt die
berechneten Gitterkonstanten aufgetragen gegen~$\cos^2{\theta}$. Der lineare Fit
der Form~$a(\theta)=m\cdot\cos^2{\theta}+b$ liefert die Parameter
%
\begin{align}
  m&=\SI{-0.07(1)}{\angstrom} \\
  \shortintertext{und}
  b&=\SI{5.677(8)}{\angstrom}.
\end{align}

\begin{figure}[h]
  \centering
  \caption{Gemessene und berechnete Größen zur Untersuchung von Probe 8.
  Bei dem ersten Tabelleneintrag handelt es sich um den berechneten Mittelwert der Doppelringstrukur.
  Es wird eine Diamant- oder eine Zinkblende-Struktur vermutet.}
  \begin{tabular}{S[table-format=3.1(1)]
                  S[table-format=2.2(1)]
                  S[table-format=1.3(3)]
                  S[table-format=1.2(2)]
                  S[table-format=2]
                  S[table-format=1.3]
                  S[table-format=1.2(2)]}
    \toprule
    {$r[\si{\milli\metre}]$} & {$\theta[\si{\degree}]$} & {$d[\si{\angstrom}]$} & {$d_1/d_i$} & {$N$} & {$\sqrt{N_i/N_1}$} & {$a[\si{\angstrom}]$} \\
    \midrule
     27.5(10) & 13.75(50) & 3.2(1)   & 1.00(5) &  3 & 1     & 5.6(2)  \\
     46.0(10) & 23.00(50) & 1.97(4)  & 1.64(7) &  8 & 1.633 & 5.6(1)  \\
     54.0(10) & 27.00(50) & 1.70(3)  & 1.91(8) & 11 & 1.915 & 5.64(10)\\
     66.5(10) & 33.25(50) & 1.41(2)  & 2.31(9) & 16 & 2.309 & 5.63(7) \\
     73.0(10) & 36.50(50) & 1.30(2)  & 2.50(9) & 19 & 2.517 & 5.65(7) \\
     84.0(10) & 42.00(50) & 1.15(1)  & 2.8(1)  & 24 & 2.828 & 5.65(5) \\
     90.5(10) & 45.25(50) & 1.086(9) & 2.9(1)  & 27 & 3.000 & 5.64(5) \\
    101.0(10) & 50.50(50) & 1.000(7) & 3.2(1)  & 32 & 3.266 & 5.66(4) \\
    107.5(10) & 53.75(50) & 0.957(6) & 3.4(1)  & 35 & 3.416 & 5.66(4) \\
    119.0(10) & 59.50(50) & 0.895(4) & 3.6(1)  & 40 & 3.651 & 5.66(3) \\
    126.5(10) & 63.25(50) & 0.864(4) & 3.8(1)  & 43 & 3.786 & 5.66(2) \\
    141.5(10) & 70.75(50) & 0.817(2) & 4.0(1)  & 48 & 4.000 & 5.66(2) \\
    153.0(10) & 76.50(50) & 0.793(2) & 4.1(1)  & 51 & 4.123 & 5.67(1) \\
    \bottomrule
  \end{tabular}
  \label{tab:probe8}
\end{figure}

\begin{figure}[htb]
  \centering
  \includegraphics[width=0.8\textwidth]{build/plot_probe8_2_2.pdf}
  \caption{Linearer Fit der berechneten Gitterkonstanten~$a$
  gegen~$\cos^2\theta$ für die untersuchte Probe 8.}
  \label{fig:probe8}
\end{figure}

\subsection{Untersuchung von Salz 2}

Die Untersuchung von Salz 2 liefert die in Tabelle~\ref{tab:salz2}
aufgeführten Ergebnisse. Die Auswertung ist aufgrund der hohen Schwärzung des Fotofilms erschwert.
Es zeigt sich, dass die beobachteten Beugungsreflexe
am besten durch eine Cäsiumchlorid-Struktur erklärt werden können. Zwar fehlt
der zur [222]-Ebene gehörige Reflex, doch kann dies auch aufgrund hoher
Schwärzung dem schwer auszuwertenden Fotofilm geschuldet sein.
Abbildung~\ref{fig:salz2} zeigt die berechneten Gitterkonstanten aufgetragen
gegen~$\cos^2{\theta}$. Der lineare Fit der
Form~$a(\theta)=m\cdot\cos^2{\theta}+b$ liefert die Parameter
%
\begin{align}
  m&=\SI{0.05(5)}{\angstrom} \\
  \shortintertext{und}
  b&=\SI{4.11(3)}{\angstrom}. % 4.113558768731562+-0.027585537949311442
\end{align}

\begin{figure}[h]
  \centering
  \caption{Gemessene und untersuchte Größen zur Untersuchung von Salz 2. Es wird
  ein sc-Gitter oder eine Cäsiumchlorid-Struktur vermutet.}
  \begin{tabular}{S[table-format=3(1)]
                  S[table-format=2.1(1)]
                  S[table-format=1.3(3)]
                  S[table-format=1.2(2)]
                  S[table-format=2]
                  S[table-format=1.3]
                  S[table-format=1.2(2)]}
    \toprule
    {$r[\si{\milli\metre}]$} & {$\theta[\si{\degree}]$} & {$d[\si{\angstrom}]$} & {$d_1/d_i$} & {$N$} & {$\sqrt{N_i/N_1}$} & {$a[\si{\angstrom}]$} \\
    \midrule
     31(1) & 15.5(5) & 2.89(9)  & 1.00(4) &  2 & 1     & 4.1(1)  \\
     45(1) & 22.5(5) & 2.02(4)  & 1.43(5) &  4 & 1.414 & 4.03(8) \\
     55(1) & 27.5(5) & 1.67(3)  & 1.73(6) &  6 & 1.732 & 4.09(7) \\
     64(1) & 32.0(5) & 1.46(2)  & 1.98(7) &  8 & 2.000 & 4.12(6) \\
     73(1) & 36.5(5) & 1.30(1)  & 2.23(7) & 10 & 2.236 & 4.10(5) \\
     89(1) & 44.5(5) & 1.10(1)  & 2.62(9) & 14 & 2.646 & 4.12(4) \\
    102(1) & 51.0(5) & 0.993(7) & 2.91(9) & 16 & 2.828 & 3.97(3) \\
    106(1) & 53.0(5) & 0.966(6) & 3.0(1)  & 18 & 3.000 & 4.10(3) \\
    114(1) & 57.0(5) & 0.920(5) & 3.1(1)  & 20 & 3.162 & 4.11(2) \\
    123(1) & 61.5(5) & 0.878(4) & 3.3(1)  & 22 & 3.317 & 4.12(2) \\
    133(1) & 66.5(5) & 0.841(3) & 3.4(1)  & 24 & 3.464 & 4.12(2) \\
    145(1) & 72.5(5) & 0.809(2) & 3.6(1)  & 26 & 3.606 & 4.12(1) \\
    \bottomrule
  \end{tabular}
  \label{tab:salz2}
\end{figure}

\begin{figure}[htb]
  \centering
  \includegraphics[width=0.8\textwidth]{build/plot_salt2_2.pdf}
  \caption{Linearer Fit der berechneten Gitterkonstanten~$a$
  gegen~$\cos^2\theta$ für das untersuchte Salz 2.}
  \label{fig:salz2}
\end{figure}
