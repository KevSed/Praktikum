\section{Theorie}
\label{sec:theorie}
%
\subsection{Die klassische Theorie der Molwärme}
%
Betrachtet man das System des Festkörpers innerhalb der klassischen Physik, so
ergibt sich für die Molwärme der Zusammenhang nach Dulong-Petit. Dazu wird
angenommen, dass die Schwinger in den Gittern des Festkörpers räumlich fest sind
und somit in \num{3}~Freiheitsgraden schwingen können. Die Schwingungen dieser
Atome sind dabei harmonische Schwingungen. Die mittlere kinetische Energie eines
harmonischen Oszillators entspricht der mittleren potentiellen Energie. Pro
Freiheitsgrad besitzt jedes Atom im Mittel die
Energie~$\sfrac{1}{2}k_{\mathrm{B}}T$, sodass sich für alle Feiheitsgrade und
die beiden Energieformen eine mittlere Energie von
%
\begin{equation}
  \langle u\rangle=2\cdot 3\cdot\frac{1}{2}k_{\mathrm{B}}T
\end{equation}
%
pro Atom ergibt. Für ein Mol an Atomen im Festkörper ergibt sich über die
passende thermodynamische Relation die material- und temperaturunabhängige Größe~$C_{\mathrm{V}}=3R$.
%
\begin{equation}
  C_{\mathrm{V}}=\left(\frac{\partial}{\partial T}U\cdot N_{\mathrm{L}}\right)_{\mathrm{V}}=3R
  \label{eq:CV}
\end{equation}
%
Dieses Ergebnis deckt sich mit den experimentellen Beobachtungen allerdings
lediglich für sehr hohe Temperaturen oder für Festkörper mit sehr hohen molaren
Massen. Daher ist hier ein anderes theoretisches Modell sinnvoll.
%
\subsection{Die Theorie nach Einstein}
%
Das Modell nach Einstein etwa berücksichtigt die Quantisierung der Energie. So
wird hier angenommen, dass Energie zwischen den einzelnen Atomen im Festkörper
nur in Vielfachen von~$\hbar\omega$ ausgetauscht werden kann. $\omega$ ist dabei
die als einheitlich angenommene bei den Schwingern im Kristall vorliegende
Frequenz. Die Wahrscheinlichkeit, dass nun ein Oszillator bei gegebener
Temperatur~$T$ im Gleichgewicht mit der Umgebung die Energie~$n\hbar\omega$
besitzt, ist dabei nach Boltzmann verteilt.
%
\begin{equation}
  W(n)=e^{-\frac{n\hbar\omega}{k_{\mathrm{B}}T}}
\end{equation}
%
Die Summation über alle Energien gewichtet mit ihrer Wahrscheinlichkeit ergibt
eine mittlere Energie pro Atom von
%
\begin{equation}
  \langle u\rangle_\text{Einstein}=\frac{\hbar\omega}{e^{\sfrac{\hbar\omega}{k_{\mathrm{B}}T}}-1}.
\end{equation}
%
Diese liegt unterhalb der klassisch vorhergesagten Energie. Über die oben
bereits genannte thermodynamische Relation ergibt sich eine Molwärme, die bei
hohen Temperaturen das im Experiment beobachtete asymptotische Streben
gegen~$3R$ zeigt und bei niedrigen Temperaturen die Abnahme der Molwärme
modelliert. Allerdings liegt diese Modellation im Bereich niedriger Temperaturen
noch nicht in der Nähe der experimentell bestimmten Werte.
%
\begin{equation}
  C_{\mathrm{V}}=3R\frac{1}{T^2}\frac{\hbar^2\omega^2}{k_{\mathrm{B}}^2}\frac{e^{\frac{\hbar\omega}{k_{\mathrm{B}}T}}}{(e^{\frac{\hbar\omega}{k_{\mathrm{B}}T}}-1)^2}
\end{equation}
%
\subsection{Die Theorie nach Debye}
%
Das Modell nach Debye berücksichtigt nun, dass die Frequenzen der
Eigenschwingungen der Atome im Festkörper nicht mehr einheitlich sind, sondern
nach einer sogenannten Spektralverteilung~$Z(\omega)$ verteilt sind. Dies
spiegelt eine deutlich realitätsnähere Annahme wider. Das Modell berücksichtigt
allerdings nicht die Frequenz- sowie die Richtungsabhängigkeit der
Phasengeschwindigkeit einer elastischen Welle im Kristall und die Rolle der
Leitungselektronen, welche allerdings erst bei sehr niedrigen Temperaturen
nennenswert wird. Ein Kristall von endlichen Ausmessungen aus~$N_{\mathrm{L}}$
Atomen besitzt nur endlich viele verschiedene Schwingungsfrequenzen,
nämlich~$3N_{\mathrm{L}}$ verschiedene. Daher muss das Integral über die
Spektralverteilung auf diesen Wert konvergieren. Dies ist nur möglich, wenn es
eine endliche Grenzfrequenz $\omega_{\mathrm{D}}$ gibt. Es gilt also
%
\begin{equation}
  \int_0^{\omega_{\mathrm{D}}}Z(\omega)\upd\omega=3N_{\mathrm{L}}.
  \label{eq:konv}
\end{equation}
%
Die Grenzfrequenz liegt je nach Berücksichtigung der Phasengeschwindigkeiten bei
%
\begin{equation}
  \omega_{\mathrm{D}}^3=\frac{6\pi^2v^3N_{\mathrm{L}}}{L^3}\quad\text{oder}\quad\omega_{\mathrm{D}}^3=\frac{18\pi^2N_{\mathrm{L}}}{L^3}\frac{v_{\mathrm{long}}^3v_{\mathrm{trans}}^3}{v_{\mathrm{trans}}^3+2v_{\mathrm{long}}^3}.
  \label{eq:omega_D}
\end{equation}
%
Die Dichte der Eigenschwingungen lässt sich auf dieselbe Weise allgemein als
%
\begin{equation}
  Z(\omega)\mathup{d}\omega=\frac{3L^3}{2\pi^2v^3}\omega^2\mathup{d}\omega\quad\text{oder}\quad Z(\omega)\mathup{d}\omega=\frac{l^3\omega^2}{2\pi^2}\frac{v_{\mathrm{trans}}^3+2v_{\mathrm{long}}^3}{v_{\mathrm{long}}^3v_{\mathrm{trans}}^3}\mathup{d}\omega
  \label{eq:Z}
\end{equation}
%
beschreiben. Aus den Gleichungen~\eqref{eq:omega_D} und~\eqref{eq:Z} folgt für
die Frequenzverteilung~$Z(\omega)$
%
\begin{equation}
  Z(\omega)\mathup{d}\omega=\frac{9N_{\mathrm{L}}}{\omega_{\mathrm{D}}^3}\omega^2\mathup{d}\omega.
\end{equation}
%
Es ergibt sich somit für die Molwärme nach Debye der Zusammenhang
%
\begin{equation}
  C_{\mathrm{V}}=9R\left(\frac{T}{\theta_{\mathrm{D}}}\right)^3\int_0^{\frac{\theta_{\mathrm{D}}}{T}}\frac{x^4e^x}{(e^x-1)^2}\mathup{d}x.
  \label{eq:debyetemperatur}
\end{equation}
%
Aus der endlichen Debye-Frequenz lässt sich sofort auch eine Debye-Temperatur
herleiten, ab welcher der Festkörper einer klassischen Beschreibung schon gut
entspricht. Die Debye-Temperatur ist
als~$\theta_{\mathrm{D}}=\sfrac{\hbar\omega_{\mathrm{D}}}{k_{\mathrm{B}}}$
definiert und eine materialspezifische Größe. Die oben in Gleichung~\eqref{eq:Z}
beschriebene Funktion allerdings stellt nun eine universell gültige Definition
der Molwärme dar. Sie zeigt für große Temperaturen das beobachtete asymptotische
Verhalten, beschreibt bei geringen Temperaturen allerdings das deutlich näher an
den experimentellen Werten liegende~\enquote{$T^3$-Gesetz}.
