\section{Diskussion}
\label{sec:diskussion}
%
Der Literaturwert für die Debye-Temperatur
beträgt~$\theta_{\mathrm{D,lit}}=\SI{343}{\kelvin}$~\cite{kittel}.
Es zeigt sich, dass dieser nur um ungefähr~\SI{3}{\percent}
von~$\theta_{\mathrm{D,2}}$ abweicht, welcher mit Hilfe theoretischer
Überlegungen aus Forderung~\eqref{eq:konv} bestimmt wurde.
Viel größer ist hingegen die Abweichung des im Versuch gemessenen
Wertes~$\theta_{\mathrm{D,1}}$ vom Literaturwert. Diese beträgt
rund~\SI{11.5}{\percent} und könnte dadurch erklärt werden, dass der
Temperaturgradient zwischen Rezipient und Probe zwischenzeitlich so groß war,
dass die Messung gestört wurde. So kann zum Beispiel Wärmestrahlung zwischen
Rezipient und Probe ungewollt Wärme zu- oder abgeführt haben. Ganz allgemein
kann die Messgenauigkeit durch Verbesserung der Methodik erhöht werden, zum
Beispiel durch eine automatisierte Auslese des Strom- bzw. Spannungsmessgerätes.
