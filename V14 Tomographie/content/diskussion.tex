\section{Diskussion}
\label{sec:diskussion}

Die Auswertung der beiden Würfel mit homogenen Materialverteilungen ergab die in
Tabelle~\ref{tab:2und3} aufgeführten Absorptionskoeffizienten. Vergleicht man 
diese mit den Literaturwerten in Tabelle~\ref{tab:lit}, so ergeben sich sehr 
eindeutige Zuordnungen zu den aufgeführten Materialien. Würfel~2 stimmt mit 
einem bestimmten Koeffizienten von $\mu_2 = \SI{1.3646(9)}{\per\centi\meter}$ 
sehr gut mit einer Zusammensetzung aus Blei überein. Die Abweichung beträgt
etwa $\SI{10}{\percent}$. Würfel~3 wies nach Messung einen 
Absorptionskoeffizienten von $\mu_3 = \SI{0.611(4)}{\per\centi\meter}$ auf.
Dieser stimmt wiederum sehr gut mit einer Zusammensetzung aus Messing überein. 
Die Abweichung beträgt hierbei nur etwa $\SI{0.5}{\percent}$.

\begin{table}[htb]
  \centering
  \caption{Absorptionskoeffizienten einiger Metalle. Die Werte folgen aus den Dichten und Absorptionskoeffizienten der einzelnen Elemente~\cite{koeff}.}
  \begin{tabular}{c
                  S[table-format=1.3]
									S[table-format=2.2]
									S[table-format=1.3]}
    \toprule
    {Material} & {$\sigma$, $\si{\centi\meter\squared\per\gram}$} & {$\rho$, $\si{\gram\per\centi\meter^{3}}$} & {$\mu$, $\si{\per\centi\meter}$} \\
		\midrule
    Blei & 0.110 & 11.34 & 1.245 \\
    Messing & 0.073 & 8.41 & 0.614 \\
	Eisen & 0.073 & 7.86 & 0.574 \\
	Aluminium & 0.075 & 2.71 & 0.203 \\
	Delrin & 0.082 & 1.42 & 0.116 \\
    \bottomrule
  \end{tabular}
  \label{tab:lit}
\end{table}

Die Messwerte für Würfel~5 lassen auf die folgende Zusammensetzung aus 
Teilwürfeln schließen.

\begin{table}[htb]
  \centering
  \caption{Aus den verschiedenen Absorptionskoeffizienten bestimmte Zusammensetzung der Teilwürfel von Würfel 5.}
  \begin{tabular}{c|
                  S[table-format=1.4(1)]
                  c}
    \toprule
    {Teilwürfel} & {Absorptionskoeffizient $\mu$, $\si{\per\centi\meter}$} & {Material} \\
	\midrule
    1 &  1.3646(9) & Blei \\
    2 &  0.611(4)  & Blei \\
    3 &  0.611(4)  & Blei \\
    4 &  0.611(4)  & Blei \\
    5 &  0.611(4)  & Blei \\
    6 &  0.611(4)  & Blei \\
    7 &  0.611(4)  & Blei \\
    8 &  0.611(4)  & Blei \\
    9 &  0.611(4)  & Blei \\ 
    \bottomrule
  \end{tabular}
  \label{tab:ergebnisse5}
\end{table}