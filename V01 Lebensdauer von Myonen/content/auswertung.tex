\section{Auswertung}
\label{sec:auswertung}

Die in der Auswertung verwendeten Mittelwerte mehrfach gemessener Größen sind gemäß der Gleichung
%
\begin{equation}
    \bar{x}=\frac{1}{n}\sum_{i=1}^n x_i
    \label{eq:mittelwert}
\end{equation}
%
bestimmt.
Die Standardabweichung des Mittelwertes ergibt sich dabei zu
%
\begin{equation}
    \symup{\Delta}\bar{x}=\sqrt{\frac{1}{n(n-1)}\sum_{i=1}^n\left(x_i-\bar{x}\right)^2}.
    \label{eq:standardabweichung}
\end{equation}
%
Resultiert eine Größe über eine Gleichung aus zwei anderen fehlerbehafteten Größen, so berechnet sich der Gesamtfehler nach der Gaußschen Fehlerfortpflanzung zu
%
\begin{equation}
    \symup{\Delta}f(x_1,x_2,...,x_n)=\sqrt{\left(\frac{\partial f}{\partial x_1}\symup{\Delta}x_1\right)^2+\left(\frac{\partial f}{\partial x_2}\symup{\Delta}x_2\right)^2+ \dotsb +\left(\frac{\partial f}{\partial x_n}\symup{\Delta}x_n\right)^2}.
    \label{eq:fehlerfortpflanzung}
\end{equation}
%
Alle in der Auswertung angegebenen Größen sind stets auf die erste signifikante Stelle des Fehlers gerundet.
Setzt sich eine Größe über mehrere Schritte aus anderen Größen zusammen, so wird erst am Ende gerundet, um Fehler zu vermeiden.
Zur Auswertung wird die Programmiersprache \texttt{python (Version 3.4.1)}
mit den Bibliothekserweiterungen \texttt{numpy}, \texttt{scipy} und \texttt{matplotlib} zur Erstellung der Grafiken und linearen Regressionen verwendet.


Die aufgenommenen Messwerte sind mit einer Poisson-Unsicherheit behaftet. Bei~$N$
erfassten Spannungsimpulsen wird ein Fehler von~$\sqrt{N}$ angenommen.

\subsection{Justierung und Überprüfung der Koinzidenzapparatur}

Die zwischen die Sekundärelektronenvervielfacher und die Koinzidenzapparatur
geschalteten Diskriminatoren liefern ab einer nicht näher bestimmten eingehenden
Spannungsschwelle~\SI{20}{\nano\second} breite Spannungsimpulse mit einer festen
Höhe von~\SI{1.3}{\volt}.

Zur Bestimmung der optimalen Verzögerung wird in einer Messreihe die Zählrate in
Abhängigkeit der Verzögerungszeit bestimmt. Die Messwerte sind in
Tabelle~\ref{tab:verzoegerung} aufgeführt und in
Abbildung~\ref{fig:verzoegerung} dargestellt. Es zeigt sich, dass eine relative
Verzögerung von~$T_{symup{VZ}}=\SI{5.5}{\nano\second}$ zu einer maximalen
Zählrate führt. Im Folgenden wird diese Verzögerungszeit verwendet.

\begin{table}[htb]
  \centering
  \caption{Messwerte zur Bestimmung der optimalen Verzögerungszeit.}
  \begin{tabular}{S[table-format = 2.1] S[table-format = 3(2)] |
                  S[table-format = 2.1] S[table-format = 3(2)] }
    \toprule
    {$t_{\symup{VZ}}$ in \si{\nano\second}} & {Impulse in \SI{20}{\second}} & {$t_{\symup{VZ}}$ in \si{\nano\second}} & {Impulse in \SI{20}{\second}} \\
    \midrule
    -10.0 & 586(24) &  4.0 & 646(25) \\
     -6.0 & 593(24) &  4.5 & 621(25) \\
     -5.0 & 579(24) &  5.0 & 616(25) \\
     -4.0 & 599(24) &  5.5 & 668(26) \\
     -3.0 & 605(24) &  6.0 & 627(25) \\
     -2.0 & 599(24) &  6.5 & 620(25) \\
     -1.0 & 562(24) &  7.0 & 639(25) \\
      0.0 & 620(25) &  7.5 & 597(24) \\
      1.0 & 577(24) &  8.0 & 582(24) \\
      2.0 & 577(24) &  8.5 & 568(24) \\
      2.5 & 574(24) &  9.0 & 550(23) \\
      3.0 & 604(25) & 10.0 & 527(23) \\
      3.5 & 558(24) &      &         \\
    \bottomrule
  \end{tabular}
  \label{tab:verzoegerung}
\end{table}

\begin{figure}[htb]
  \centering
  \includegraphics[width=0.8\textwidth]{analysis/verzoegerung.pdf}
  \caption{Messwerte zur Bestimmung der optimalen Verzögerungszeit.}
  \label{fig:verzoegerung}
\end{figure}

Vor der Koinzidenzschaltung werden~\num{4718} bzw.~\num{4662} Impulse
in~\SI{60}{\second} gemessen. Dies entspricht einer gemittelten Rate
von~\num{78.2(5)} Impulsen pro Sekunde. Am Ausgang der Koinzidenzapparatur
werden~\num{1752} Impulse in~\SI{60}{\second} gemessen, was einer Rate
von~\num{29.2} Impulsen pro Sekunde entspricht. Die Rate wird somit um
rund~\SI{62}{\percent} durch die Koinzidenzschaltung gedämpft.

\subsection{Zeitkalibrierung des Vielkanalanalysators}

Zur Kalibrierung der Zeitachse des Vielkanalanalysators wird eine Messreihe mit
Hilfe eines Doppelimpulsgenerators durchgeführt. Der Doppelimpulsgenerator
liefert mit einer hohen Rate je zwei Spannungsimpulse, deren zeitlicher Abstand
über einen Kodierschalter festgelegt werden kann. Die Zeitdifferenzen, sowie die
jeweils ansprechenden Kanäle sind in Tabelle~\ref{tab:kalibrierung} aufgeführt.
Es zeigt sich, dass für eine Zeitdifferenz häufig nicht nur ein Kanal, sondern
auch die direkt benachbarten Kanäle ansprechen. Dies ist vermutlich auf
Störungen in der Übertragungsleitung des Vielkanalanalysators zurückzuführen.
Um dennoch einem Kanal eine bestimmte Zeitdifferenz zuordnen zu können, wird der
Mittelwert~$\bar{x}_i$ aller bei einer bestimmten
Zeitdifferenz~$\symup{\Delta}t_i$ ansprechenden Kanäle bestimmt, wobei die
relative Häufigkeit, bei der ein bestimmter Kanal angesprochen hat, als
Gewichtung in den Mittelwert einfließt:
%
\begin{equation}
  \bar{x}_i=\frac{1}{N_i}\sum_{j=1}^{N_i}n_{ij}x_{ij}
  \label{eq:kanal_mittelwert}
\end{equation}
%
Hierbei ist~$N_i$ die Gesamtzahl an registrierten Ereignissen für eine bestimmte
Zeitdifferenz. $n_{ij}$ ist die Anzahl der Ereignisse für einen bestimmten
Kanal~$x_{ij}$. Weiterhin ergibt sich für jeden Mittelwert eine
Standardabweichung~$\symup{\Delta}\bar{x}_i$:
%
\begin{equation}
  \symup{\Delta}\bar{x}_i=\sqrt{\frac{1}{N_i}\sum_{j=1}^{N_i}n_{ij}\left(x_{ij}-\bar{x}_i\right)^2}
  \label{eq:kanal_standardabweichung}
\end{equation}
%
Um nun allen Kanälen eine eindeutige Zeitdifferenz zuordnen zu können, wird eine
lineare Regression der gemessenen Stützstellen durchgeführt. Diese liefert die
Gerade
%
\begin{equation}
  \symup{\Delta}t=\SI{0.045417(9)}{\micro\second}\cdot c+\SI{0.002(1)}{\micro\second}
  \label{eq:kalibrierung}
\end{equation}
%
wobei~$\symup{c}$ für die Nummer des Kanals steht.
Abbildung~\ref{fig:kalibrierung} zeigt die aufgenommen Messwerte sowie die
berechnete Regressionsgerade.

\begin{table}[htb]
  \centering
  \caption{Messwerte zur Zeikalibrierung des Vielkanalanalysators.}
  \begin{tabular}{S[table-format = 2.1] S[table-format = 3.1(1)]}
    \toprule
    {$\symup{\Delta}t$ in \si{\micro\second}} & {Zugeordneter Kanal} \\
    \midrule
    1.0 &  22.0(0) \\
    2.0 &  44.0(0) \\
    3.0 &  66.0(0) \\
    4.0 &  87.9(4) \\
    5.0 & 110.0(7) \\
    6.0 & 132.1(8) \\
    7.0 & 154.1(8) \\
    8.0 & 176.1(8) \\
    9.0 & 198.1(8) \\
    \bottomrule
  \end{tabular}
  \label{tab:kalibrierung}
\end{table}

\begin{figure}[htb]
  \centering
  \includegraphics[width=0.8\textwidth]{analysis/kalibrierung.pdf}
  \caption{Messung zur Zeitkalibrierung des Vielkanalanalysators. Neben den
  Messwerten ist auch die berechnete Regressionsgerade aufgetragen.}
  \label{fig:kalibrierung}
\end{figure}

\subsection{Abschätzung des Untergrundes}

Insgesamt wurden~$N_{\symup{S}}=\num{4179917(2044)}$ Startimpulse bei einer
Gesamtmesszeit von~\SI{152016}{\second} registriert. Die daraus resultierende
Rate beträgt
%
\begin{equation}
  \langle N_{\symup{S}}\rangle=\SI{27.50(1)}{\per\second}.
  \label{eq:rate_start}
\end{equation}
%
Die Anzahl der Untergrundkandidaten bestimmt sich aus der Wahrscheinlichkeit,
dass innerhalb der Suchzeit~$T_{\symup{S}}=\SI{20}{\micro\second}$, die durch
ein Myon gestartet wurde, ein weiteres Myon vom Detektor registriert wird. Dabei
folgt die Wahrscheinlichkeit einer Poissonverteilung. Somit ergibt sich die
Anzahl~$N_{\symup{F}}$ der registrierten Fehlereignisse zu
%
\begin{equation}
  N_{\symup{F}}(k=1)=N_{\symup{S}}\cdot\frac{\left(T_{\symup{S}}\langle N_{\symup{S}}\rangle\right)^k}{k!}\exp\left(T_{\symup{S}}\langle N_{\symup{S}}\rangle\right)=\num{2297(1)}.
  \label{eq:untergrund_gesamt}
\end{equation}
%
Unter der Annahme, dass sich der erwartete Untergrund statistisch auf alle
Kanäle gleichverteilt, lässt sich der Untergrund pro Kanal bestimmen. Dabei
werden bei der Berechnung die Kanäle~\num{441} bis~\num{511} nicht berücksichtigt, da
diese Kanäle gemäß Gleichung~\ref{eq:kalibrierung} Zeitdifferenzen erfassen, die
die Suchzeit~$T_{\symup{S}}$ übersteigen. Umgerechnet auf die
verbleibenden~\num{440} Kanäle ergibt sich der erwartete mittlere Untergrund pro
Kanal zu
%
\begin{equation}
  U_0=\num{5.221(3)}
  \label{eq:untergrund_kanal}
\end{equation}
%
Der Fehler von~$U_0$ wird im Folgenden aufgrund der geringen Größe
vernachlässigt.

\subsection{Bestimmung der mittleren Lebensdauer eines Myons}

Die vom Vielkanalanalysator erfassten Messwerte sind im Anhang aufgelistet.
Abbildung~\ref{fig:spektrum1} zeigt die Messwerte in einem Histogramm. Um nun
die mittlere Lebensdauer~$\tau$ eines Myons aus der Zerfallskonstanten~$\lambda$
bestimmen zu können, wird eine Funktion der Form
%
\begin{equation}
  N(c)=N_0\exp\left(-\lambda_cc\right)+U_0
  \label{eq:fitfunktion}
\end{equation}
%
an die Messwerte angepasst. Hierbei ist~$c$ wieder die Kanalnummer. Für die
Anpassung werden die Kanäle~\num{0} bis~\num{2}, sowie die Kanäle~\num{12}
bis~\num{18} ausgeschlossen, da sie zu stark vom exponentiellen Verlauf
abweichen und somit als Folge eines fehlerhaften Versuchsaufbaus interpretiert
werden können. Eine weitere Diskussion der Abweichungen findet sich in
Abschnitt~\ref{sec:diskussion}. Desweiteren werden erneut die Kanäle~\num{441}
bis~\num{511} aus oben genannten Gründen ausgeschlossen. Die Anpassung der
verbleibenden Messwerte an die Funktion in Gleichung~\ref{eq:fitfunktion}
liefert die Parameter
%
\begin{align}
  N_0&=\num{727(8)} \\
  \lambda_c&=\num{0.02144(18)} \\
  U_0&=\num{3.76(24)}
  \label{eq:fitparameter}
\end{align}
%
$\lambda_c$ ist in diesem Fall noch nicht die gesuchte
Zerfallskostante~$\lambda$. Diese ergibt sich erst durch Division
von~$\lambda_c$ mit der Steigung der Regressionsgeraden in
Gleichung~\ref{eq:kalibrierung}. Die mittlere Lebensdauer~$\tau$ eines Myons
ergibt sich als Inverses der Zerfallskonstante~$\lambda$ zu
%
\begin{equation}
  \tau=\SI{2.118(17)}{\micro\second}
  \label{eq:ergebnis}
\end{equation}
%
Die Abbildungen~\ref{fig:spektrum2} und~\ref{fig:spektrum3} zeigen die
Ausgleichsfunktion zusammen mit den Messwerten.

\begin{figure}[htb]
  \centering
  \includegraphics[width=0.8\textwidth]{analysis/spektrum1.pdf}
  \caption{Häufigkeitsverteilung der gemessenen Zerfallszeiten, ausgedrückt
  durch die entsprechenden Kanäle des Vielkanalanalysators.}
  \label{fig:spektrum1}
\end{figure}

\begin{figure}[htb]
  \centering
  \includegraphics[width=0.8\textwidth]{analysis/spektrum2.pdf}
  \caption{Messwerte zur Berechnung der mittleren Lebensdauer von Myonen und
  Ausgleichsfunktion.}
  \label{fig:spektrum2}
\end{figure}

\begin{figure}[htb]
  \centering
  \includegraphics[width=0.8\textwidth]{analysis/spektrum3.pdf}
  \caption{Messwerte zur Berechnung der mittleren Lebensdauer von Myonen und
  Ausgleichsfunktion (logarithmisch skaliert).}
  \label{fig:spektrum3}
\end{figure}
