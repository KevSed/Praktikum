\section{Theorie}
\label{sec:theorie}

Das Standardmodell der Teilchenphysik beschreibt die Eigenschaften und
Wechselwirkungen der fundamentalen Elementarteilchen. Das Standardmodell
unterscheidet zwischen den Quarks, die an der elektroschwachen und der starken
Wechselwirkung teilnehmen und den Leptonen, die nur von der elektroschwachen
Wechselwirkung beeinflusst werden. In diesem Versuch werden Myonen betrachtet,
die zu den Leptonen gehören. Die sechs bekannten Leptonen werden in drei
Generationen zu je zwei Teilchen unterteilt. Jede Generation beinhaltet ein
geladenes und ein ungeladenes Lepton (Neutrinos). Das Myon ist das geladene
Lepton der zweiten Generation. Kosmische Myonen entstehen ...


Nur die erste Leptongeneration ist stabil.
