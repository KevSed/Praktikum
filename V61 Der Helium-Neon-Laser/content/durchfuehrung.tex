\section{Durchführung}
\label{sec:durchführung}

Der Versuchsaufbau besteht aus einer optischen Schiene, auf der ein Laserrohr
und zwei hochreflektierende Resonatorspiegel angebracht sind. Ein Spiegel dient
als Auskoppelspiegel, durch den der Strahl den Resonator teilweise
verlassen kann. Das mit dem Helium-Neon-Gasgemisch befüllte Laserrohr wird
durch zwei Brewster-Fenster begrenzt. Parallel zur Einfallsebene polarisiertes
Licht passiert die Fenster nahezu verlustfrei, während senkrecht zur
Einfallsebene polarisiertes Licht zum Teil reflektiert wird. Somit wird im Laser
eine Polarisationsrichtung bevorzugt. Des Weiteren stehen ein bereits
installierter Justierlaser, sowie weitere grundlegende optische Komponenten wie
eine Photodiode, ein Einzelspalt und ein variabler Polarisator zur Verfügung.

In einem ersten Schritt wird der Helium-Neon-Laser (HeNe-Laser) justiert und zum
lasen gebracht. Dazu werden die Resonatorspiegel und das Laserrohr einzeln auf
die optische Schiene gestellt und mit Hilfe eines Justierlasers so ausgerichtet,
dass die Rückreflexe der einzelnen Komponenten mit der Strahlachse zur Deckung
kommen. Damit ist sichergestellt, dass das ganze System entlang der optischen
Achse ausgerichtet ist. In einem zweiten Schritt wird der Justierlaser
ausgeschaltet und das Laserrohr eingeschaltet. Der Strom am Laserrohr wird
auf~\SI{6.5}{\ampere} eingestellt. Unter Beachtung der Stabilitätsbedingung,
das heißt bei richtigem Abstand der Resonatorspiegel zueinander, beginnt der
Laser zu lasen.

Im Folgenden werden verschiedene voneinander unabhängige Untersuchung gemacht,
um Charakteristika des HeNe-Lasers zu bestimmen:

Zur Überprüfung der Stabilitätsbedingung~\eqref{eq:bedingung} werden die
Resonatorspiegel mit maximal großem Abstand zueinander positioniert und so
justiert, dass der Laser eine maximale Leistung abgibt. Die Messung der
Leistung in Form eines Photstromes erfolgt mit Hilfe einer am Ende des
Strahlgangs befindlichen Photodiode. Anschließend wird der Abstand der
Resonatorspiegel variiert und der Photostrom gemessen. Die Überprüfung der
Stabilitätsbedingung erfolgt zunächst mit zwei konkaven Resonatorspiegeln z
je~\SI{1400}{\milli\metre} Radius. In einer zweiten Messung wird einer dieser
Spiegel durch einen planaren Spiegel ausgetauscht.

In einer weiteren Untersuchung werden zwei TEM-Moden des Lasers vermessen.
Dazu wird hinter den Auskoppelspiegel eine Linse positioniert, die den
austretenden Strahl verbreitert. Mit der Photodiode wird anschließend die
Intensität der $\text{TEM}_{00}$-Grundmode in Abhängigkeit der senkrechten
Auslenkung vermessen. Durch das Einbringen eines dünnen Haars in den Strahl
zwischen Laserrohr und Auskoppelspiegel kann die $\text{TEM}_{10}$-Mode
vermessen werden. Das Haar dient als sogenannte Modenblende, mit deren Hilfe die
Grundmode unterdrückt werden kann.

In einer dritten Messung wird die Polarisation des Lasers bestimmt. Dazu wird
zwischen Auskoppelspiegel und Photodiode ein Polarisationsfilter gebracht und
die Strahlintensität in Abhängigkeit des eingestellten Winkels des Filters
gemessen.

Zuletzt wird die Wellenlänge der emittierten Laserstrahlung bestimmt. Dazu wird
ein Gitter hinter dem Auskoppelspiegel positioniert. Das entstehende
Beugungsbild wird auf einen Schirm geworfen. Es werden die Abstände der
verschiedenen Beugungsmaxima zueinander und der Abstand vom Gitter zum Schirm
gemessen.
